%%%%%%%%%%%%%%%%%%%%%%%%%%%%%%%%%%%%%%%%%%%%%%%%%%%%%%%%%%%%%%%%%
%%%%%%%%%%%%%%%%%%%%%%%%%%%%%%%%%%%%%%%%%%%%%%%%%%%%%%%%%%%%%%%%%
% 中文~4.20~翻译:
% 5.2.5-5.2.11 gprsnl@bbs.ctex
% 其他章节     zpxing@bbs.ctex  email: zpxing at gmail dot com
%%%%%%%%%%%%%%%%%%%%%%%%%%%%%%%%%%%%%%%%%%%%%%%%%%%%%%%%%%%%%%%%%
\setcounter{chapter}{4}
\newcommand{\graphicscompanion}{\emph{The \LaTeX{} Graphics Companion}~\cite{graphicscompanion}}
\newcommand{\hobby}{\emph{A User's Manual for MetaPost}~\cite{metapost}}
\newcommand{\hoenig}{\emph{\TeX{} Unbound}~\cite{unbound}}
\newcommand{\graphicsinlatex}{\emph{Graphics in \LaTeXe{}}~\cite{ursoswald}}

%\chapter{Producing Mathematical Graphics}
%\label{chap:graphics}
\chapter{数学图形}
\label{chap:graphics}

%\begin{intro}
%Most people use \LaTeX\ for typesetting their text. But as the non content and
%structure oriented approach to authoring is so convenient, \LaTeX\ also offers a,
%if somewhat restricted, possibility for producing graphical output from textual
%descriptions. Furthermore, quite a number of \LaTeX\ extensions have been created
%in order to overcome these restrictions. In this section, you will learn about a
%few of them.
%\end{intro}
\begin{intro}
大部分人使用 \LaTeX 来排版文本内容。 因其不面向内容和结构的特点给写作提供了巨大的方便,
我们还可以有办法从文本描述生成图形输出。此外,大量的 \LaTeX 扩展
被开发出来以克服种种限制。 在本节中,我们将学习其中的一些。
\end{intro}
%\section{Overview}
\section{概述}

%The \ei{picture} environment allows programming pictures directly in
%\LaTeX. A detailed
%description can be found in the \manual. On the one hand, there are rather
%severe constraints, as the slopes of line segments as well as the radii of
%circles are restricted to a narrow choice of values.  On the other hand, the
%\ei{picture} environment of \LaTeXe\ brings with it the \ci{qbezier}
%command, ``\texttt{q}'' meaning ``quadratic''.  Many frequently used curves
%such as circles, ellipses, or catenaries can be satisfactorily approximated
%by quadratic B\'ezier curves, although this may require some mathematical
%toil. If, in addition, a programming language like Java is used to generate
%\ci{qbezier} blocks of \LaTeX\ input files, the \ei{picture} environment
%becomes quite powerful.

\ei{picture} 环境可以在 \LaTeX{} 里直接设计图形。详细的介绍请参考 \manual。
一方面,这种方法有严重的局限性,比如线段的斜率和圆的半径只能在一个很小的范围内取值。
另一方面, \LaTeXe 的 \ei{picture} 环境提供了 \ci{qbezier} 命令,
``\texttt{q}'' 表示 ``quadratic''。许多常用的曲线如圆、椭圆、或者悬链线都
可以用二次 B\'ezier 曲线得到令人满意的近似,虽然这可能需要一些辛苦的数学准备。
另外,如果有一种编程语言如 Java 能用来生成 \LaTeX 源文档的 \ci{qbezier} 模块,
\ei{picture} 环境会更强大。

%Although programming pictures directly in \LaTeX\ is severely
%restricted, and often rather tiresome, there are still reasons for
%doing so. The documents thus produced are ``small'' with respect to
%bytes, and there are no additional graphics files to be dragged
%along.
%
%Packages like \pai{epic} and \pai{eepic} (described, for instance,
%in \companion), or \pai{pstricks} help to eliminate the restrictions
%hampering the original \ei{picture} environment, and greatly
%strengthen the graphical power of \LaTeX.
%
%While the former two packages just enhance the \ei{picture}
%environment, the \pai{pstricks} package has its own drawing
%environment, \ei{pspicture}. The power of \pai{pstricks} stems from
%the fact that this package makes extensive use of \PSi{}
%possibilities. In addition, numerous packages have been written for
%specific purposes. One of them is \texorpdfstring{\Xy}{Xy}-pic,
%described at the end of this chapter. A wide variety of these
%packages is described in detail in \graphicscompanion{} (not to be
%confused with \companion).

虽然直接在 \LaTeX 里设计图形的方法有严重的局限性而且通常比较繁琐,
但它还是很有用的。这份文档就是用它才变得体积很小,不需要插入额外的图片。

一些宏包,如 \pai{epic} 和 \pai{eepic}(\companion 里有介绍),或者
 \pai{pstricks} 可以排除 \ei{picture} 环境的局限,并大大地增强了 \LaTeX 的图形功能。

跟前两个宏包只是加强了 \ei{picture} 环境不同,\pai{pstricks} 宏包有自己的绘图环境,
\ei{pspicture}。 \pai{pstricks} 的强大之处在于它广泛应用了 \PSi{}。
另外,许多宏包可以用来处理专门的问题。其一是 \texorpdfstring{\Xy}{Xy}-pic,
本章最后会讲到它。
\graphicscompanion{} (勿与 \companion 混淆)里详细介绍了大量的宏包.
%
%Perhaps the most powerful graphical tool related with \LaTeX\ is \texttt{MetaPost}, the twin of
%Donald E. Knuth's \texttt{METAFONT}. \texttt{MetaPost} has the very powerful and
%mathematically sophisticated programming language of \texttt{METAFONT}. Contrary to \texttt{METAFONT},
%which generates bitmaps, \texttt{MetaPost} generates encapsulated \PSi{} files,
%which can be imported in \LaTeX. For an introduction, see \hobby, or the tutorial on \cite{ursoswald}.

\LaTeX 最强大的图形工具可能是 \texttt{MetaPost}, Donald E.
Knuth 编写的 \texttt{METAFONT} 的孪生兄弟。
\texttt{MetaPost} 使用非常强大的数学编程语言: \texttt{METAFONT}。
与 \texttt{METAFONT} 生成点阵图片不同,\texttt{MetaPost} 生成的是封装的 \PSi{} 文件,
可以导入 \LaTeX 中。其介绍可以看 \hobby,或者 \cite{ursoswald}。

%
%A very thorough discussion of \LaTeX{} and \TeX{} strategies for graphics (and fonts) can
%be found in \hoenig.

关于 \LaTeX{} 和 \TeX{} 图形(以及字体)支持方法的详细讨论请参考 \hoenig。

%\section{The \texttt{picture} Environment}
%\secby{Urs Oswald}{osurs@bluewin.ch}
\section{\texttt{picture} 环境}
\secby{Urs Oswald}{osurs@bluewin.ch}


%\subsection{Basic Commands}
\subsection{基本命令}

%A \ei{picture} environment\footnote{Believe it or not, the picture environment works out of the
%box, with standard \LaTeXe{} no package loading necessary.} is created with one of the two commands
一个 \ei{picture} 环境\footnote{信不信由你,picture 环境仅需标准的 \LaTeXe{},“开箱即用”,无需载入宏包。}可以用下面两个命令中的一个来创建
\begin{lscommand}
\ci{begin}\verb|{picture}(|$x,y$\verb|)|\ldots\ci{end}\verb|{picture}|
\end{lscommand}
\noindent 或者
\begin{lscommand}
\ci{begin}\verb|{picture}(|$x,y$\verb|)(|$x_0,y_0$\verb|)|\ldots\ci{end}\verb|{picture}|
\end{lscommand}
%The numbers $x,\,y,\,x_0,\,y_0$ refer to \ci{unitlength}, which can be reset any time
%(but not within a \ei{picture} environment) with a command such as
数字 $x,\,y,\,x_0,\,y_0$ 是相对于 \ci{unitlength} 而言的,任何时候(除了在 \ei{picture} 环境之内以外),都可以
使用命令如
\begin{lscommand}
\ci{setlength}\verb|{|\ci{unitlength}\verb|}{1.2cm}|
\end{lscommand}
%The default value of \ci{unitlength} is \texttt{1pt}. The first
%pair, $(x,y)$, effects the reservation, within the document, of
%rectangular space for the picture. The optional second pair,
%$(x_0,y_0)$, assigns arbitrary coordinates to the bottom left corner
%of the reserved rectangle.
\noindent
来改变。\ci{unitlength} 的默认值是 \texttt{1 pt}。第一个数对,
$(x,y)$, 在文档中为图形保留一个矩形的区域。可选的第二个数对,
$(x_0,y_0)$,为矩形左下角指派任意的坐标。

%Most drawing commands have one of the two forms
大多数的绘图命令是下面两种格式之一
\begin{lscommand}
\ci{put}\verb|(|$x,y$\verb|){|\emph{object}\verb|}|
\end{lscommand}
\noindent 或者
\begin{lscommand}
\ci{multiput}\verb|(|$x,y$\verb|)(|$\Delta x,\Delta
y$\verb|){|$n$\verb|}{|\emph{object}\verb|}|\end{lscommand}
%B\'ezier curves are an exception. They are drawn with the command
B\'ezier 曲线是一个例外。 它们需要用命令
\begin{lscommand}
\ci{qbezier}\verb|(|$x_1,y_1$\verb|)(|$x_2,y_2$\verb|)(|$x_3,y_3$\verb|)|
\end{lscommand}
\noindent 来画。
\newpage

%\subsection{Line Segments}
\subsection{线段}
\begin{example}
\setlength{\unitlength}{5cm}
\begin{picture}(1,1)
  \put(0,0){\line(0,1){1}}
  \put(0,0){\line(1,0){1}}
  \put(0,0){\line(1,1){1}}
  \put(0,0){\line(1,2){.5}}
  \put(0,0){\line(1,3){.3333}}
  \put(0,0){\line(1,4){.25}}
  \put(0,0){\line(1,5){.2}}
  \put(0,0){\line(1,6){.1667}}
  \put(0,0){\line(2,1){1}}
  \put(0,0){\line(2,3){.6667}}
  \put(0,0){\line(2,5){.4}}
  \put(0,0){\line(3,1){1}}
  \put(0,0){\line(3,2){1}}
  \put(0,0){\line(3,4){.75}}
  \put(0,0){\line(3,5){.6}}
  \put(0,0){\line(4,1){1}}
  \put(0,0){\line(4,3){1}}
  \put(0,0){\line(4,5){.8}}
  \put(0,0){\line(5,1){1}}
  \put(0,0){\line(5,2){1}}
  \put(0,0){\line(5,3){1}}
  \put(0,0){\line(5,4){1}}
  \put(0,0){\line(5,6){.8333}}
  \put(0,0){\line(6,1){1}}
  \put(0,0){\line(6,5){1}}
\end{picture}
\end{example}
%Line segments are drawn with the command
线段用命令
\begin{lscommand}
\ci{put}\verb|(|$x,y$\verb|){|\ci{line}\verb|(|$x_1,y_1$\verb|){|$length$\verb|}}|
\end{lscommand}
%Line segments are drawn with the command
\noindent 来画。 命令 \ci{line} 有两个参量:
%\begin{enumerate}
%  \item a direction vector,
%  \item a length.
%\end{enumerate}
\begin{enumerate}
  \item 一个方向向量,
  \item 一个长度。
\end{enumerate}
%The components of the direction vector are restricted to the integers
方向向量需由以下整数构成
\[
  -6,\,-5,\,\ldots,\,5,\,6,
\]
%and they have to be coprime (no common divisor except 1). The figure illustrates all
%25 possible slope values in the first quadrant. The length is relative to \ci{unitlength}.
%The length argument is the vertical coordinate in the case of a vertical line segment, the
%horizontal coordinate in all other cases.
而且它们需要互质(除 1 以外,没有公约数),图形显示了第一象限中所有 25 个可能的斜率值。
长度是相对于 \ci{unitlength} 来说的。长度的参量当一个垂直线段时是垂直坐标,其他情况都是水平坐标。

%\subsection{Arrows}
\subsection{箭头}

\begin{example}
\setlength{\unitlength}{0.75mm}
\begin{picture}(60,40)
  \put(30,20){\vector(1,0){30}}
  \put(30,20){\vector(4,1){20}}
  \put(30,20){\vector(3,1){25}}
  \put(30,20){\vector(2,1){30}}
  \put(30,20){\vector(1,2){10}}
  \thicklines
  \put(30,20){\vector(-4,1){30}}
  \put(30,20){\vector(-1,4){5}}
  \thinlines
  \put(30,20){\vector(-1,-1){5}}
  \put(30,20){\vector(-1,-4){5}}
\end{picture}
\end{example}
%Arrows are drawn with the command
画箭头要用命令
\begin{lscommand}
\ci{put}\verb|(|$x,y$\verb|){|\ci{vector}\verb|(|$x_1,y_1$\verb|){|$length$\verb|}}|
\end{lscommand}
%For arrows, the components of the direction vector are even more narrowly restricted than
%for line segments, namely to the integers
箭头的方向向量元素比线段的限制更严格,需由以下整数构成
\[
  -4,\,-3,\,\ldots,\,3,\,4.
\]
%Components also have to be coprime (no common divisor except 1). Notice the effect  of the
%\ci{thicklines} command on the two arrows pointing to the upper left.
而且需要互质(除 1 以外,没有公约数)。注意命令 \ci{thicklines} 对指向左上方的两个箭头产生的效果。

%\subsection{Circles}
\subsection{圆}

\begin{example}
\setlength{\unitlength}{1mm}
\begin{picture}(60, 40)
  \put(20,30){\circle{1}}
  \put(20,30){\circle{2}}
  \put(20,30){\circle{4}}
  \put(20,30){\circle{8}}
  \put(20,30){\circle{16}}
  \put(20,30){\circle{32}}

  \put(40,30){\circle{1}}
  \put(40,30){\circle{2}}
  \put(40,30){\circle{3}}
  \put(40,30){\circle{4}}
  \put(40,30){\circle{5}}
  \put(40,30){\circle{6}}
  \put(40,30){\circle{7}}
  \put(40,30){\circle{8}}
  \put(40,30){\circle{9}}
  \put(40,30){\circle{10}}
  \put(40,30){\circle{11}}
  \put(40,30){\circle{12}}
  \put(40,30){\circle{13}}
  \put(40,30){\circle{14}}

  \put(15,10){\circle*{1}}
  \put(20,10){\circle*{2}}
  \put(25,10){\circle*{3}}
  \put(30,10){\circle*{4}}
  \put(35,10){\circle*{5}}
\end{picture}
\end{example}
%The command
命令
\begin{lscommand}
  \ci{put}\verb|(|$x,y$\verb|){|\ci{circle}\verb|{|\emph{diameter}\verb|}}|
\end{lscommand}
%\noindent draws a circle with center $(x,y)$ and diameter (not radius) \emph{diameter}.
%The \ei{picture} environment only admits diameters up to approximately 14\,mm,
%and even below this limit, not all diameters are possible. The \ci{circle*}
%command produces disks (filled circles).
\noindent
画了一个圆心在 $(x,y)$ 直径(不是半径)为 \emph{diameter} 的圆。
\ei{picture} 环境只允许直径最大是 14\,mm, 而且即使在这个限制之下,
也不是所有的直径都可获得。命令 \ci{circle*} 生成圆盘 (填充的圆形)。


%As in the case of line segments, one may have to resort to additional packages,
%such as \pai{eepic} or \pai{pstricks}.
%For a thorough description of these packages, see \graphicscompanion.

跟线段的情况一样,你可能需要其他宏包的帮助,比如 \pai{eepic} 或者 \pai{pstricks}。
这些宏包的详细说明请参考 \graphicscompanion。

%There is also a possibility within the
%\ei{picture} environment. If one is not afraid of doing the necessary calculations
%(or leaving them to a program), arbitrary circles and ellipses can be patched
%together from quadratic B\'ezier curves.
%See \graphicsinlatex\ for examples and Java source files.
\ei{picture} 环境还有另外一个可能。如果你不怕麻烦的必要的计算(或者交给一个程序来处理),
任意的圆和矩形都可以由二次 B\'ezier 曲线拼成。请看例子 \graphicsinlatex 以及 Java 源文件。


% \subsection{Text and Formulas}
\subsection{文本与公式}

\begin{example}
\setlength{\unitlength}{0.8cm}
\begin{picture}(6,5)
  \thicklines
  \put(1,0.5){\line(2,1){3}}
  \put(4,2){\line(-2,1){2}}
  \put(2,3){\line(-2,-5){1}}
  \put(0.7,0.3){$A$}
  \put(4.05,1.9){$B$}
  \put(1.7,2.95){$C$}
  \put(3.1,2.5){$a$}
  \put(1.3,1.7){$b$}
  \put(2.5,1.05){$c$}
  \put(0.3,4){$F=
    \sqrt{s(s-a)(s-b)(s-c)}$}
  \put(3.5,0.4){$\displaystyle
    s:=\frac{a+b+c}{2}$}
\end{picture}
\end{example}
% As this example shows, text and formulas can be written into a \ei{picture} environment with
% the \ci{put} command in the usual way.
如本例所示,文本与公式可以使用 \ci{put} 命令按照正常方式在 \ei{picture} 环境中使
用。

% \subsection{\ci{multiput} and \ci{linethickness}}
\subsection{\ci{multiput}~与~\ci{linethickness}}

\begin{example}
\setlength{\unitlength}{2mm}
\begin{picture}(30,20)
  \linethickness{0.075mm}
  \multiput(0,0)(1,0){26}%
    {\line(0,1){20}}
  \multiput(0,0)(0,1){21}%
    {\line(1,0){25}}
  \linethickness{0.15mm}
  \multiput(0,0)(5,0){6}%
    {\line(0,1){20}}
  \multiput(0,0)(0,5){5}%
    {\line(1,0){25}}
  \linethickness{0.3mm}
  \multiput(5,0)(10,0){2}%
    {\line(0,1){20}}
  \multiput(0,5)(0,10){2}%
    {\line(1,0){25}}
\end{picture}
\end{example}
% The command
% \begin{lscommand}
%   \ci{multiput}\verb|(|$x,y$\verb|)(|$\Delta x,\Delta y$\verb|){|$n$\verb|}{|\emph{object}\verb|}|
% \end{lscommand}
% \noindent has 4 arguments: the starting point, the translation vector from one object to the next,
% the number of objects, and the object to be drawn. The \ci{linethickness} command applies to
% horizontal and vertical line segments, but neither to oblique line segments, nor to circles.
% It does, however, apply to quadratic B\'ezier curves!
命令
\begin{lscommand}
  \ci{multiput}\verb|(|$x,y$\verb|)(|$\Delta x,\Delta y$\verb|){|$n$\verb|}{|\emph{object}\verb|}|
\end{lscommand}
\noindent
有 4 个参量:初始点,从一个对象到下一个的平移向量,对象的数目和要绘制
的对象。命令 \ci{linethickness} 可作用于水平和垂直方向的线段,但不能作用于倾斜的
线段和圆。然而,该命令可作用于二次 B\'ezier 曲线。

% \subsection{Ovals}
\subsection{椭圆}

\begin{example}
\setlength{\unitlength}{0.75cm}
\begin{picture}(6,4)
  \linethickness{0.075mm}
  \multiput(0,0)(1,0){7}%
    {\line(0,1){4}}
  \multiput(0,0)(0,1){5}%
    {\line(1,0){6}}
  \thicklines
  \put(2,3){\oval(3,1.8)}
  \thinlines
  \put(3,2){\oval(3,1.8)}
  \thicklines
  \put(2,1){\oval(3,1.8)[tl]}
  \put(4,1){\oval(3,1.8)[b]}
  \put(4,3){\oval(3,1.8)[r]}
  \put(3,1.5){\oval(1.8,0.4)}
\end{picture}
\end{example}
% The command
% \begin{lscommand}
%   \ci{put}\verb|(|$x,y$\verb|){|\ci{oval}\verb|(|$w,h$\verb|)}|
% \end{lscommand}
% \noindent or
% \begin{lscommand}
%   \ci{put}\verb|(|$x,y$\verb|){|\ci{oval}\verb|(|$w,h$\verb|)[|\emph{position}\verb|]}|
% \end{lscommand}
% \noindent produces an oval centered at $(x,y)$ and having width $w$ and height $h$. The optional
% \emph{position} arguments \texttt{b}, \texttt{t}, \texttt{l}, \texttt{r} refer to
% ``top'', ``bottom'', ``left'', ``right'', and can be combined, as the example illustrates.
命令
\begin{lscommand}
  \ci{put}\verb|(|$x,y$\verb|){|\ci{oval}\verb|(|$w,h$\verb|)}|
\end{lscommand}
\noindent 或
\begin{lscommand}
  \ci{put}\verb|(|$x,y$\verb|){|\ci{oval}\verb|(|$w,h$\verb|)[|\emph{position}\verb|]}|
\end{lscommand}
\noindent
可以产生一个中心在 $(x,y)$ 处、宽为 $w$ 高为 $h$ 的椭圆。如本例所示,可选
参量 \emph{position} 可以是 \texttt{b}, \texttt{t}, \texttt{l},
\texttt{r}, 分别
表示仅绘制椭圆的“下部”、“上部”、“左部”和“右部”,如例所示,这些参数可以进行组合。

% Line thickness can be controlled by two kinds of commands: \\
% \ci{linethickness}\verb|{|\emph{length}\verb|}|
% on the one hand, \ci{thinlines} and \ci{thicklines} on the other. While \ci{linethickness}\verb|{|\emph{length}\verb|}|
% applies only to horizontal and vertical lines (and quadratic B\'ezier curves), \ci{thinlines} and \ci{thicklines}
% apply to oblique line segments as well as to circles and ovals.
以下两类命令可以控制线宽:一类
为 \ci{linethickness}\verb|{|\emph{length}\verb|}|,另一类
为 \ci{thinlines} 与 \ci{thicklines}。命
令 \ci{linethickness}\verb|{|\emph{length}\verb|}| 仅对水平和垂直直线(及二次 B\'ezier 曲线)有作用,
\ci{thinlines} 与 \ci{thicklines} 则可以作用于倾斜的线段、圆和椭圆。


% \subsection{Multiple Use of Predefined Picture Boxes}
\subsection{重复使用预定义的图形盒子}

\begin{example}
\setlength{\unitlength}{0.5mm}
\begin{picture}(120,168)
\newsavebox{\foldera}
\savebox{\foldera}
  (40,32)[bl]{% definition
  \multiput(0,0)(0,28){2}
    {\line(1,0){40}}
  \multiput(0,0)(40,0){2}
    {\line(0,1){28}}
  \put(1,28){\oval(2,2)[tl]}
  \put(1,29){\line(1,0){5}}
  \put(9,29){\oval(6,6)[tl]}
  \put(9,32){\line(1,0){8}}
  \put(17,29){\oval(6,6)[tr]}
  \put(20,29){\line(1,0){19}}
  \put(39,28){\oval(2,2)[tr]}
}
\newsavebox{\folderb}
\savebox{\folderb}
  (40,32)[l]{%         definition
  \put(0,14){\line(1,0){8}}
  \put(8,0){\usebox{\foldera}}
}
\put(34,26){\line(0,1){102}}
\put(14,128){\usebox{\foldera}}
\multiput(34,86)(0,-37){3}
  {\usebox{\folderb}}
\end{picture}
\end{example}
% A picture box can be \emph{declared} by the command
% \begin{lscommand}
%   \ci{newsavebox}\verb|{|\emph{name}\verb|}|
% \end{lscommand}
% \noindent then \emph{defined} by
% \begin{lscommand}
%   \ci{savebox}\verb|{|\emph{name}\verb|}(|\emph{width,height}\verb|)[|\emph{position}\verb|]{|\emph{content}\verb|}|
% \end{lscommand}
% \noindent and finally arbitrarily often be \emph{drawn} by
% \begin{lscommand}
%   \ci{put}\verb|(|$x,y$\verb|)|\ci{usebox}\verb|{|\emph{name}\verb|}|
% \end{lscommand}
一个图形盒子可以使用命令
\begin{lscommand}
  \ci{newsavebox}\verb|{|\emph{name}\verb|}|
\end{lscommand}
\noindent 进行\textbf{声明},然后使用命令
\begin{lscommand}
  \ci{savebox}\verb|{|\emph{name}\verb|}(|\emph{width,height}\verb|)[|\emph{position}\verb|]{|\emph{content}\verb|}|
\end{lscommand}
\noindent 进行\textbf{定义},最后使用命令
\begin{lscommand}
  \ci{put}\verb|(|$x,y$\verb|)|\ci{usebox}\verb|{|\emph{name}\verb|}|
\end{lscommand}
\noindent 进行任意次数的重复\textbf{绘制}。

% The optional \emph{position} parameter has the effect of defining the
% `anchor point' of the savebox. In the example it is set to \texttt{bl} which
% puts the anchor point into the bottom left corner of the savebox. The other
% position specifiers are \texttt{t}op and \texttt{r}ight.
可选参数 \emph{position} 的作用是定义图形存放盒子的“锚点”。在本例中该参数被设置
为 \texttt{bl},从而将锚点设置为图形存放盒子的左下角。其他的位置描述
有 \texttt{t} 和 \texttt{r},分别表示“上”和“右”。

% The \emph{name} argument refers to a \LaTeX{} storage bin and therefore is
% of a command nature (which accounts for the backslashes in the current
% example). Boxed pictures can be nested: In this example, \ci{foldera} is
% used within the definition of \ci{folderb}.
参量 \emph{name} 指明了 \LaTeX{} 存储槽,揭示了其命令本质(在本例中指反斜线)。图
形盒子可以嵌套:在本例中,\ci{foldera} 被用在了 \ci{folderb} 的定义中。

% The \ci{oval} command had to be used as the \ci{line} command does not work if
% the segment length is less than about 3\,mm.
由于命令 \ci{line} 在线段长度小于大约 3\,mm 的时候不能正常工作,所以必须使用命令 \ci{oval}。

% \subsection{Quadratic B\'ezier Curves}
\subsection{二次~B\'ezier~曲线}

\begin{example}
\setlength{\unitlength}{0.8cm}
\begin{picture}(6,4)
  \linethickness{0.075mm}
  \multiput(0,0)(1,0){7}
    {\line(0,1){4}}
  \multiput(0,0)(0,1){5}
    {\line(1,0){6}}
  \thicklines
  \put(0.5,0.5){\line(1,5){0.5}}
  \put(1,3){\line(4,1){2}}
  \qbezier(0.5,0.5)(1,3)(3,3.5)
  \thinlines
  \put(2.5,2){\line(2,-1){3}}
  \put(5.5,0.5){\line(-1,5){0.5}}
  \linethickness{1mm}
  \qbezier(2.5,2)(5.5,0.5)(5,3)
  \thinlines
  \qbezier(4,2)(4,3)(3,3)
  \qbezier(3,3)(2,3)(2,2)
  \qbezier(2,2)(2,1)(3,1)
  \qbezier(3,1)(4,1)(4,2)
\end{picture}
\end{example}
% As this example illustrates, splitting up a circle into 4 quadratic B\'ezier curves
% is not satisfactory. At least 8 are needed. The figure again shows the effect of
% the \ci{linethickness} command on horizontal or vertical lines, and of the
% \ci{thinlines} and the \ci{thicklines} commands on oblique line segments. It also
% shows that both kinds of commands affect quadratic B\'ezier curves, each command
% overriding all previous ones.
如本例所示,将圆分割为 4 条二次 B\'ezier 曲线的效果不能令人满意,至少需要 8 条。该图
再一次展示了命令 \ci{linethickness} 对水平或垂直直线以及命
令 \ci{thinlines} 和 \ci{thicklines} 对倾斜线段的影响。该例同时显示:这两类命令都
会影响二次 B\'ezier 曲线,每一条命令都会覆盖以前所有命令。

% Let $P_1=(x_1,\,y_1),\,P_2=(x_2,\,y_2)$ denote the end points, and $m_1,\,m_2$ the
% respective slopes, of a quadratic B\'ezier curve. The intermediate control point
% $S=(x,\,y)$ is then given by the equations
令 $P_1=(x_1,\,y_1),\,P_2=(x_2,\,y_2)$ 和 $m_1,\,m_2$ 分别表示一条二次 B\'ezier 曲线
的两个端点及其对应斜率。中间控制点 $S=(x,\,y)$ 则由下述方程给出
\begin{equation} \label{zwischenpunkt}
  \left\{
    \begin{array}{rcl}
      x & = & \displaystyle \frac{m_2 x_2-m_1x_1-(y_2-y_1)}{m_2-m_1}, \\
      y & = & y_i+m_i(x-x_i)\qquad (i=1,\,2).
    \end{array}
  \right.
\end{equation}
% \noindent See \graphicsinlatex\ for a Java program which generates
% the necessary \ci{qbezier} command line.
\noindent
关于生成必要的 \ci{qbezier} 命令的 Java 程序参见 \graphicsinlatex。

% \subsection{Catenary}
\subsection{悬链线}

\begin{example}
\setlength{\unitlength}{1cm}
\begin{picture}(4.3,3.6)(-2.5,-0.25)
\put(-2,0){\vector(1,0){4.4}}
\put(2.45,-.05){$x$}
\put(0,0){\vector(0,1){3.2}}
\put(0,3.35){\makebox(0,0){$y$}}
\qbezier(0.0,0.0)(1.2384,0.0)
  (2.0,2.7622)
\qbezier(0.0,0.0)(-1.2384,0.0)
  (-2.0,2.7622)
\linethickness{.075mm}
\multiput(-2,0)(1,0){5}
  {\line(0,1){3}}
\multiput(-2,0)(0,1){4}
  {\line(1,0){4}}
\linethickness{.2mm}
\put( .3,.12763){\line(1,0){.4}}
\put(.5,-.07237){\line(0,1){.4}}
\put(-.7,.12763){\line(1,0){.4}}
\put(-.5,-.07237){\line(0,1){.4}}
\put(.8,.54308){\line(1,0){.4}}
\put(1,.34308){\line(0,1){.4}}
\put(-1.2,.54308){\line(1,0){.4}}
\put(-1,.34308){\line(0,1){.4}}
\put(1.3,1.35241){\line(1,0){.4}}
\put(1.5,1.15241){\line(0,1){.4}}
\put(-1.7,1.35241){\line(1,0){.4}}
\put(-1.5,1.15241){\line(0,1){.4}}
\put(-2.5,-0.25){\circle*{0.2}}
\end{picture}
\end{example}

% In this figure, each symmetric half of the catenary $y=\cosh x -1$ is approximated by a quadratic
% B\'ezier curve. The right half of the curve ends in the point \((2,\,2.7622)\), the slope there having the value
% \(m=3.6269\). Using again equation (\ref{zwischenpunkt}), we can
% calculate the intermediate control points. They turn out to be $(1.2384,\,0)$ and $(-1.2384,\,0)$.
% The crosses indicate points of the \emph{real} catenary. The error is barely noticeable, being less
% than one percent.
在本图中,悬链线 $y=\cosh x
-1$ 对称的两半由二次 B\'ezier 曲线分别近似地绘成。曲线的右
半部分终止于点 \((2,\,2.7622)\),对应的斜率为 \(m=3.6269\)。再次使用公
式 (\ref{zwischenpunkt}),我们可以计算中间控制点。计算结果
为 $(1.2384,\,0)$ 和 $(-1.2384,\,0)$。图中的十字为{\textbf
真正}的悬链线上的点。误差 小于百分之一,很难被发现。

% This example points out the use of the optional argument of the \\
% \verb|\begin{picture}| command.
% The picture is defined in convenient ``mathematical'' coordinates, whereas by the command
% \begin{lscommand}
%   \ci{begin}\verb|{picture}(4.3,3.6)(-2.5,-0.25)|
% \end{lscommand}
% \noindent its lower left corner (marked by the black disk) is assigned the
% coordinates $(-2.5,-0.25)$.
该例指出了命令 \verb|\begin{picture}| 的可选参数的用法。该图通过使用命令
\begin{lscommand}
  \ci{begin}\verb|{picture}(4.3,3.6)(-2.5,-0.25)|
\end{lscommand}
\noindent 定义了方便的“数学”坐标:左下角(由黑色圆点标出)坐标是
$(-2.5,-0.25)$。

% \subsection{Rapidity in the Special Theory of Relativity}
\subsection{坐标的相对性}

\begin{example}
\setlength{\unitlength}{0.8cm}
\begin{picture}(6,4)(-3,-2)
  \put(-2.5,0){\vector(1,0){5}}
  \put(2.7,-0.1){$\chi$}
  \put(0,-1.5){\vector(0,1){3}}
  \multiput(-2.5,1)(0.4,0){13}
    {\line(1,0){0.2}}
  \multiput(-2.5,-1)(0.4,0){13}
    {\line(1,0){0.2}}
  \put(0.2,1.4)
    {$\beta=v/c=\tanh\chi$}
  \qbezier(0,0)(0.8853,0.8853)
    (2,0.9640)
  \qbezier(0,0)(-0.8853,-0.8853)
    (-2,-0.9640)
  \put(-3,-2){\circle*{0.2}}
\end{picture}
\end{example}
% The control points of the two B\'ezier curves were calculated with formulas (\ref{zwischenpunkt}).
% The positive branch is determined by $P_1=(0,\,0),\,m_1=1$ and $P_2=(2,\,\tanh 2),\,m_2=1/\cosh^2 2$.
% Again, the picture is defined in mathematically convenient coordinates, and the lower left corner
% is assigned the mathematical coordinates $(-3,-2)$ (black disk).
公式 (\ref{zwischenpunkt}) 给出了两条 B\'ezier 曲线的控制点。正向分支
由 $P_1=(0,\,0),\,m_1=1$ 和 $P_2=(2,\,\tanh 2),\,m_2=1/\cosh^2 2$ 确定。与前例相
同,本图也定义了在数学上方便的坐标,左下角的坐标是 $(-3,-2)$ (黑点)。

%\section{\texorpdfstring{\Xy}{Xy}-pic}
%\secby{Alberto Manuel Brand\~ao Sim\~oes}{albie@alfarrabio.di.uminho.pt}
\section{\texorpdfstring{\Xy}{Xy}-pic}
\secby{Alberto Manuel Brand\~ao Sim\~oes}{albie@alfarrabio.di.uminho.pt}

%\pai{xy} is a special package for drawing diagrams. To use it,
%simply add the following line to the preamble of your document:
%\begin{lscommand}
%\verb|\usepackage[|\emph{options}\verb|]{xy}|
%\end{lscommand}
%\emph{options} is a list of functions from \Xy-pic you want to
%load. These options are primarily useful when debugging the package.  I recommend
%you pass the \verb!all! option, making \LaTeX{} load all the \Xy{} commands.

\pai{xy} 是绘制流程图的专用宏包。要想使用它,只需在导言区加上:
\begin{lscommand}
\verb|\usepackage[|\emph{options}\verb|]{xy}|
\end{lscommand}
\emph{options} 列出你需要载入的 \Xy-pic 的选项。这些选项基本上被用于调试这个宏包的使用。
建议你使用 \verb!all!,可以让 \LaTeX{} 载入 \Xy{} 的所有命令。

%\Xy-pic diagrams are drawn over a matrix-oriented canvas, where
%each diagram element is placed in a matrix slot:
\Xy-pic 流程图被绘制在一幅以矩阵定位的画布上,每一个流程图元素被放在矩阵的一个单元中:
\begin{example}
\begin{displaymath}
\xymatrix{A & B \\
          C & D }
\end{displaymath}
\end{example}
%The \ci{xymatrix} command must be used in math mode. Here, we
%specified two lines and two columns. To make this matrix a diagram we
%just add directed arrows using the \ci{ar} command.
命令 \ci{xymatrix} 必须置于数学模式中。这里,我们设定了一个两行两列的矩阵。
为了画出流程,我们只需要使用命令 \ci{ar} 增加带方向的箭头即可。
\begin{example}
\begin{displaymath}
\xymatrix{ A \ar[r] & B \ar[d] \\
           D \ar[u] & C \ar[l] }
\end{displaymath}
\end{example}
%The arrow command is placed on the origin cell for the arrow. The
%arguments are the direction the arrow should point to (\texttt{u}p,
%\texttt{d}own, \texttt{r}ight and \texttt{l}eft).
箭头命令要放在其出发的那个单元里。参量是箭头的方向 (\texttt{u}:上,
\texttt{d}:下, \texttt{r}:右以及 \texttt{l}:左).


\begin{example}
\begin{displaymath}
\xymatrix{
  A \ar[d] \ar[dr] \ar[r] & B \\
  D                       & C }
\end{displaymath}
\end{example}
%To make diagonals, just use more than one direction. In
%fact, you can repeat directions to make bigger arrows.
要画对角线,可以指定不只一个方向参量。实际上,你还可以重复同一个方向来得到更大的箭头。
\begin{example}
\begin{displaymath}
\xymatrix{
 A \ar[d] \ar[dr] \ar[drr] &&\\
 B                   & C & D }
\end{displaymath}
\end{example}

%We can draw even more interesting diagrams by adding
%labels to the arrows. To do this, we use the common superscript and
%subscript operators.
我们还可以绘制一些更有趣的流程图,给箭头加上标签,只需要使用普通的上标和下标。
\begin{example}
\begin{displaymath}
\xymatrix{
  A \ar[r]^f \ar[d]_g &
             B \ar[d]^{g'} \\
  D \ar[r]_{f'}       & C }
\end{displaymath}
\end{example}

%As shown, you use these operators as in math mode. The only
%difference is that that superscript means ``on top of the arrow,''
%and subscript means ``under the arrow.'' There is a third operator, the vertical bar: \verb+|+
%It causes text to be placed \emph{in} the arrow.
如图所示,就像数学模式里一样使用上下标。唯一的区别在于:上标表示放在 “箭头的上方”,
下标表示放在“箭头的下方”。 把文本放到箭头上可以用 \verb+|+。
\begin{example}
\begin{displaymath}
\xymatrix{
  A \ar[r]|f \ar[d]|g &
             B \ar[d]|{g'} \\
  D \ar[r]|{f'}       & C }
\end{displaymath}
\end{example}

%To draw an arrow with a hole in it, use \verb!\ar[...]|\hole!.
绘制空心箭头的命令是 \verb!\ar[...]|\hole!。

%In some situations, it is important to distinguish between different types of
%arrows. This can be done by putting labels on them, or changing their appearance:
某些情况下,需要区分不同类型的箭头。可以给它们标上标签,或者使用不同的外观来实现:

\begin{example}
\shorthandoff{"}
\begin{displaymath}
\xymatrix{
\bullet\ar@{->}[rr] && \bullet\\
\bullet\ar@{.<}[rr] && \bullet\\
\bullet\ar@{~)}[rr] && \bullet\\
\bullet\ar@{=(}[rr] && \bullet\\
\bullet\ar@{~/}[rr] && \bullet\\
\bullet\ar@{^{(}->}[rr] &&
                       \bullet\\
\bullet\ar@2{->}[rr] && \bullet\\
\bullet\ar@3{->}[rr] && \bullet\\
\bullet\ar@{=+}[rr]  && \bullet
}
\end{displaymath}
\shorthandon{"}
\end{example}

%Notice the difference between the following two diagrams:
注意下面两幅流程图的区别:

\begin{example}
\begin{displaymath}
\xymatrix{
 \bullet \ar[r]
         \ar@{.>}[r] &
 \bullet
}
\end{displaymath}
\end{example}

\begin{example}
\begin{displaymath}
\xymatrix{
 \bullet \ar@/^/[r]
         \ar@/_/@{.>}[r] &
 \bullet
}
\end{displaymath}
\end{example}

%The modifiers between the slashes define how the curves are drawn.
%\Xy-pic offers many ways to influence the drawing of curves;
%for more information, check \Xy-pic documentation.
两条斜线间的修饰元素决定了曲线应该如何被画出。
\Xy-pi 提供了很多办法来改变曲线的形状;更详细的内容请参考 \Xy-pic 的文档。

% \begin{example}
% \begin{lscommand}
% \ci{dum}
% \end{lscommand}
% \end{example}
