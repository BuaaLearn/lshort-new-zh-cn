\chapter{文档元素}\label{chap:elements}

\begin{intro}
在知道了如何输入文字后,我们将在本章了解一个结构化的文档所依赖的各种元素——章节、目录、列表、图表、交叉引用、脚注等等。
\end{intro}

\section{章节和目录}\label{sec:secs}

\subsection{章节标题}\label{subsec:secs}

\cmdindex{section,subsection,subsubsection,paragraph,subparagraph}
一篇结构化的、条理清晰文档一定是层次分明的,通过不同的命令分割为章、节、小节。\LaTeX\ 的三个标准文档类 \cls{article}、\cls{report} 和 \cls{book}%
\footnote{千万注意是\textbf{标准文档类},其它文档类可能没有定义或只定义了一部分,如 \cls{beamer} 或 \cls{moderncv} 等。}%
提供了一系列命令,用以划分章节、生成章节标题并自动编号:
\begin{command}
\cmd{section}\marg{title} \quad
\cmd{subsection}\marg{title} \quad
\cmd{subsubsection}\marg{title} \\
\cmd{paragraph}\marg{title} \quad
\cmd{subparagraph}\marg{title}
\end{command}

\cmdindex{part}
\cmd{part} 命令用以将整个文档分割为大的分块,但不影响 \cmd{section} 等的编号:
\begin{command}
\cmd{part}\marg{title}
\end{command}

\cmdindex{chapter}
\clsindex{book,report}
\cls{book} 和 \cls{report} 提供了章一级的结构:
\begin{command}
\cmd{chapter}\marg{title}
\end{command}

上述命令除了生成带编号的标题之外,还向目录中添加条目,并影响页眉页脚的内容(详见 \ref{sec:pagestyle} 节)。每个命令有两种变体:
\begin{itemize}
  \item 带可选参数的变体:\cmd{section}\oarg{short title}\marg{title}\par
  标题使用 \Arg{title} 参数,在目录和页眉页脚中使用 \Arg{short title} 参数;
  \item 带星号的变体:\cmd{section*}\marg{title}\par
  标题不带编号,也不生成目录项和页眉页脚。
\end{itemize}

\subsection{目录}\label{sec:toc}

\cmdindex{tableofcontents}
在 \LaTeX\ 中生成目录非常容易,只需在合适的地方使用命令:
\begin{command}
\cmd{tableofcontents}
\end{command}

正确生成目录项,一般需要多次编译源代码。

\cmdindex{addcontentsline}
有时我们使用了 \cmd{chapter*} 或 \cmd{section*} 这样不生成目录项的命令,而又想手动生成该章节的目录,可以在标题命令后面使用:
\begin{command}
\cmd{addcontentsline}\marg*{toc}\marg{level}\marg{title}
\end{command}

其中 \Arg{level} 为章节层次 \texttt{chapter} 或 \texttt{section} 等,\Arg{title} 为需要生成目录项的章节标题。

\subsection{文档结构的划分}\label{sec:matters}

\cmdindex{appendix}
所有标准文档类都提供了一个 \cmd{appendix} 命令将正文和附录分开%
\footnote{有的地方可能写作 \cmd{begin} \marg*{appendix} \ldots \cmd{end} \marg*{appendix} 这样,
虽然是可行的,但这种写法不规范,只要使用 \cmd{appendix} 命令就够了。},
使用 \cmd{appendix} 后,最高一级章节改为使用拉丁字母编号,从 A 开始。

\clsindex{book}
\cmdindex{frontmatter,mainmatter,backmatter}
\cls{book} 文档类还提供了前言、正文、后记结构的划分命令:
\begin{description}
  \item[\cmd{frontmatter}] 前言部分,页码为小写罗马字母格式;其后的章节不编号(但生成页眉页脚和目录项)。
  \item[\cmd{mainmatter}] 正文部分,页码改回数字格式,并从 1 开始计数;其后的章节正常编号。
  \item[\cmd{backmatter}] 后记部分,页码继续正常计数,格式不变;其后的章节不编号,类似 \cmd{frontmatter} 命令。
\end{description}

以上三个命令还可和 \cmd{appendix} 命令结合,生成有前言、正文、附录、后记四部分的文档。

\section{标题页}\label{sec:titlepage}

\cmdindex{title,author,date,today}
\LaTeX\ 支持生成简单的标题页。首先需要给定标题和作者等信息:
\begin{command}
\cmd{title}\marg{title} \quad
\cmd{author}\marg{author} \quad
\cmd{date}\marg{date}
\end{command}
其中前两个命令是必须的,\cmd{date} 命令可选。\LaTeX\ 还提供了一个 \cmd{today} 命令自动生成当前日期,
可用在 \cmd{date} 的参数里,或者别的地方。

\cmdindex{thanks}
在 \cmd{title}、\cmd{author} 等命令内可以使用 \cmd{thanks} 命令生成标题页的脚注,如:
\begin{verbatim}
\author{Mary\thanks{E-mail:*****@***.com}}
\end{verbatim}

\cmdindex{maketitle}
\pinyinindex{biaotiye}{标题页}
在信息给定后,就可以使用
\begin{command}
\cmd{maketitle}
\end{command}
生成一个简单的标题页了。\cls{article} 文档类的标题不单独成页,而 \cls{report} 和 \cls{book} 单独成页。

\section{交叉引用}\label{sec:crossref}

\cmdindex{label}
交叉引用是 \LaTeX\ 强大的自动排版功能的体现之一。在能够被交叉引用的地方,如章节、公式、图表、定理等位置使用 \cmd{label} 命令:
\begin{command}
\cmd{label}\marg{label-name}
\end{command}

\cmdindex{ref,pageref}
之后就可以在别的地方使用 \cmd{ref} 或 \cmd{pageref} 命令,分别生成交叉引用的编号和页码:
\begin{command}
\cmd{ref}\marg{label-name} \\
\cmd{pageref}\marg{label-name}
\end{command}
\begin{example}
A reference to this subsection
\label{sec:this} looks like:
``see section~\ref{sec:this} on
page~\pageref{sec:this}.''
\end{example}

生成正确的交叉引用一般也需要多次编译源代码。

\cmd{label} 命令可用于记录各种类型的交叉引用,使用位置分别为:
\begin{description}
  \item[章节标题] 在章节标题命令 \cmd{section} 等之后紧接着使用。
  \item[行间公式] 单行公式(\env{equation} 环境,见 \ref{sec:math-basics} 节)在公式内任意位置使用;
    多行公式(\pkg{amsmath} 提供的 \env{align} 环境等,见 \ref{subsec:align} 小节)在每一行公式的任意位置使用。
  \item[有序列表] 在 \env{enumerate} 环境的每个 \cmd{item} 命令之后、下一个 \cmd{item} 命令之前任意位置使用,见 \ref{subsec:lists} 小节。
  \item[图表浮动体] 在图表标题命令 \cmd{caption} 之后紧接着使用,见某小节。
  \item[定理环境] 在定理环境内部任意位置使用,见 \ref{sec:theorems} 节。
\end{description}

在使用不记编号的命令形式(\cmd{section*}、\cmd{caption*}、带可选参数的 \cmd{item} 命令等)时不要使用 \cmd{label} 命令,
否则生成的引用编号不正确。

\section{脚注}\label{sec:footnote}

\cmdindex{footnote}
使用 \cmd{footnote} 命令可以在页面底部生成一个脚注:
\begin{command}
\cmd{footnote}\marg{footnote}
\end{command}

假如我们输入以下文字和命令:
\begin{verbatim}
“天地玄黄,宇宙洪荒。日月盈昃,辰宿列张。”\footnote{语出《千字文》。}
\end{verbatim}

在正文中则为:%
“天地玄黄,宇宙洪荒。日月盈昃,辰宿列张。”\footnote{语出《千字文》。}

\cmdindex{footnotemark,footnotetext}
有些情况下(比如在表格环境、盒子内)使用 \cmd{footnote} 并不能正确生成脚注。我们可以分两步进行,
先使用 \cmd{foot\-note\-mark} 为脚注计数,再在合适的位置用 \cmd{foot\-note\-text} 生成脚注。

\section{特殊环境}\label{sec:envs}

\subsection{列表}\label{subsec:lists}

\envindex{enumerate,itemize}
\cmdindex{item}
\LaTeX\ 提供了基本的有序和无序列表环境 \env{enumerate} 和 \env{itemize},两者的用法很类似,都用 \cmd{item} 标明每个列表项。
\env{enumerate} 环境会自动对列表项编号。
\begin{command}
\cmd{begin}\marg*{enumerate} \\
\cmd{item} \ldots \\
\cmd{end}\marg*{enumerate}
\end{command}

其中 \cmd{item} 可带一个可选参数,将有序列表的计数或者无序列表的符号替换成自定义的符号。
列表可以嵌套使用,最多嵌套四层。
\begin{example}
\begin{enumerate}
  \item An item. 
  \begin{enumerate}
    \item A nested item.
    \item[*] A starred item.
    \item Another item. \label{itref}
  \end{enumerate}
  \item Go back to upper level.
  \item Reference(\ref{itref}).
\end{enumerate}
\end{example}

\begin{example}
\begin{itemize}
  \item An item.
  \begin{itemize}
    \item A nested item.
    \item[+] A `plus' item.
    \item Another item.
  \end{itemize}
  \item Go back to upper level.
\end{itemize}
\end{example}

\envindex{description}
关键字环境 \env{description} 的用法与以上两者类似,不同的是 \cmd{item} 后的可选参数用来写关键字,以粗体显示,一般是必填的:
\begin{command}
\cmd{begin}\marg*{description} \\
  \cmd{item}\oarg{item title} \ldots \\
\cmd{end}\marg*{description}
\end{command}

\begin{example}
\begin{description}
  \item[Enumerate] Numbered list.
  \item[Itemize] Non-numbered list.
\end{description}
\end{example}

默认的列表间距比较宽,\LaTeX\ 本身也难以修改,一般要用到 \pkg{enumitem} 宏包。本文不详细叙述,有兴趣的可查阅其帮助手册。

\subsection{对齐环境}\label{subsec:flush}

\envindex{center,flushleft,flushright}
\env{center}、\env{flush\-left} 和 \env{flush\-right} 环境分别用于生成居中、左对齐和右对齐的文本环境。
\begin{command}
\cmd{begin}\marg*{center}
\ldots
\cmd{end}\marg*{center}
\end{command}

\begin{example}
\begin{center}
Centered text using a
\verb|center| environment.
\end{center}
\begin{flushleft}
Left-aligned text using a
\verb|flushleft| environment.
\end{flushleft}
\begin{flushright}
Right-aligned text using a
\verb|flushright| environment.
\end{flushright}
\end{example}

\cmdindex{centering,raggedleft,raggedright}
除此之外,还可以用以下命令直接改变文字的对齐方式:
\begin{command}
\cmd{centering} \quad
\cmd{raggedright} \quad
\cmd{raggedleft}
\end{command}

\begin{example}
\centering
Centered text paragraph.

\raggedright
Left-aligned text paragraph.

\raggedleft
Right-aligned text paragraph.
\end{example}

三个命令和对应的环境经常被误用,有直接用所谓 \cmd{flushleft} 命令或者 \env{raggedright} 环境的,都是不合理的用法(即使它们可能有效)。
有一点可以将两者区分开来:\env{center} 等环境会在上下文产生一个额外间距,而 \cmd{centering} 等命令不产生,只是改变对齐方式。
比如在浮动体环境 \env{table} 或 \env{figure} 内实现居中对齐,用 \cmd{centering} 命令即可,没必要再用 \env{center} 环境。

\subsection{引用环境}\label{subsec:quote}

\envindex{quote,quotation}
\LaTeX\ 提供了两种引用的环境:\env{quote} 用于引用较短的文字,首行不缩进;\env{quotation} 用于引用若干段文字,首行缩进。
引用环境较一般文字有额外的左右缩进。
\begin{example}
Francis Bacon says:
\begin{quote}
Knowledge is power.
\end{quote}
\end{example}

\begin{example}
《木兰诗》:
\begin{quotation}
万里赴戎机,关山度若飞。
朔气传金柝,寒光照铁衣。
将军百战死,壮士十年归。

归来见天子,天子坐明堂。
策勋十二转,赏赐百千强。……
\end{quotation}
\end{example}

\envindex{verse}
\env{verse} 用于排版诗歌,与 \env{quotation} 恰好相反,\env{verse} 是首行悬挂缩进的。
\begin{example}
Rabindranath Tagore's short poem:
\begin{verse}
Beauty is truth’s smile 
when she beholds her own face in 
a perfect mirror.
\end{verse}
\end{example}

\subsection{摘要环境}\label{subsec:abstract}

\envindex{abstract}
摘要环境 \env{abstract} 只在 \cls{article} 和 \cls{report} 文档类可用,一般用于紧跟 \cmd{maketitle} 命令之后介绍文档的摘要。
如果文档类给定了 \texttt{titlepage} 选项,则单独成页;反之相当于一个小标题加一个 \env{quotation} 环境%
(双栏下相当于 \cmd{section*} 定义的一节)。

\subsection{代码环境}\label{subsec:verbatim}

\envindex{verbatim}
有时我们需要将一段代码原样转义输出,这就要用到代码环境 \env{verbatim},它以等宽字体排版代码,回车和空格也分别起到换行和空位的作用;
带星号的版本更进一步将空格显示成 \textvisiblespace 。
\begin{example}
\begin{verbatim}
#include <iostream>
int main()
{
  std::cout << 'Hello, world'
            << std::endl;
  return 0;
}
\end{verbatim}
\end{example}

\begin{example}
\begin{verbatim*}
for (int i=0; i<4; ++i)
  printf('Number %d',i);
\end{verbatim*}
\end{example}

\cmdindex{verb}
要排版简短的代码或关键字,可使用 \cmd{verb} 命令:
\begin{command}
\cmd{verb}\Arg{delim}\Arg{code}\Arg{delim}
\end{command}

\Arg{delim} 标明代码的分界位置,不能是字母、空格或星号,前后必须一致,可任意选择使得不与代码本身冲突,习惯上使用 \texttt| 符号。

同 \env{verbatim} 环境,\cmd{verb} 后也可以带一个星号,以显示空格:
\begin{example}
\verb|\LaTeX| \\
\verb+(a || b)+ \\
\verb*+(a || b)+
\end{example}

\cmd{verb} 命令对符号的处理比较复杂,一般\textbf{不能用在其它命令的参数里},否则多半会出错。

\section{表格}\label{sec:tabular}

\pinyinindex{biaoge}{表格|(}
\LaTeX\ 里排版表格不如 Word 等所见即所得的工具简便和自由,不过对于不太复杂的表格来讲,完全能够胜任。

\envindex{tabular}
\cmdindex{hline}
\index{&@\texttt\& (\textit{单元格/对齐})}
\index{\@\crcmd\ (\textit{换行})}
排版表格最基本的 \env{tabular} 环境用法为:
\begin{command}
\cmd{begin}\marg*{tabular}\marg{column-spec}\\
 \Arg{item1} \texttt\& \Arg{item2} \texttt\& \ldots\ \crcmd \\
 \cmd{hline} \\
 \Arg{item1} \texttt\& \Arg{item2} \texttt\& \ldots\ \crcmd \\
\cmd{end}\marg*{tabular}
\end{command}
其中 \Arg{column-spec} 是列格式标记,在接下来的内容将仔细介绍;\texttt\& 用来分隔单元格;
\crcmd\ 用来换行;\cmd{hline} 用来在行与行之间绘制横线。

直接使用 \env{tabular} 环境的话,会\textbf{和周围的文字混排}。\env{tabular} 环境可带一个可选参数控制垂直对齐(默认是垂直居中):
\begin{example}
\begin{tabular}{|c|}
  center-\\ aligned \\
\end{tabular},
\begin{tabular}[t]{|c|}
  top-\\ aligned \\
\end{tabular},
\begin{tabular}[b]{|c|}
  bottom-\\ aligned\\
\end{tabular} tabulars.
\end{example}

但是通常情况下我们不这么用,\env{tabular} 环境一般会放置在 \env{table} 浮动体环境中,并用 \cmd{caption} 命令加标题。

\subsection{列格式}\label{subsec:tabular-cols}

\LaTeX\ 表格中基本的列格式如下表:
\begin{center}
\begin{tabular}{lp{24em}}
 \hline
 列格式 & 说明 \\
 \hline
 \ttfamily l/c/r          & 单元格内容左对齐/居中/右对齐,不折行 \\
 \ttfamily p\marg{width}  & 单元格宽度固定为 \Arg{width},可自动折行 \\
 \ttfamily |              & 绘制竖线 \\
 \ttfamily @\marg{string} & 自定义内容 \Arg{string} \\
 \hline
\end{tabular}
\end{center}

\begin{example}
\begin{tabular}{lcr|p{6em}}
  \hline
  left & center & right 
       & par box with fixed width\\
  L    & C      & R     & P \\
 \hline
\end{tabular}
\end{example}

\texttt{@} 格式可在单元格前后插入任意的文本,但同时它也消除了单元格前后额外添加的间距。
\texttt{@} 格式可以适当使用以充当“竖线”。特别地,\texttt{@}\marg*{} 可直接用来消除单元格前后的间距:
\begin{example}
\begin{tabular}{@{} r@{:}lr @{}}
  \hline
  1  & 1 & one \\
  11 & 3 & eleven \\
  \hline
\end{tabular}
\end{example}

另外 \LaTeX\ 还提供了简便的将格式参数重复的写法 \texttt*\marg{n}\marg{column-spec},比如以下两种写法是等效的:
\begin{verbatim}
\begin{tabular}{|c|c|c|c|c|p{4em}|p{4em}|}
\begin{tabular}{|*{5}{c|}*{2}{p{4em}|}}
\end{verbatim}

\pkgindex{array}
有时需要为整列修饰格式,比如整列改变为粗体,如果每个单元格都加上 \cmd{bfseries} 命令会比较麻烦。
\pkg{array} 宏包提供了辅助格式 \texttt> 和 \texttt<,用于给列格式前后加上修饰命令:
\begin{example}
\begin{tabular}{>{\itshape}r<{*}l}
  \hline
  italic & normal \\
  column & column \\
  \hline
\end{tabular}
\end{example}

辅助格式甚至支持插入 \cmd{centering} 等命令改变 \texttt{p} 列格式的对齐方式,一般还要加额外的命令
\cmd{array\-back\-slash} 以免出错\footnote{\cmd{centering} 等对齐命令会破坏表格环境里 \crcmd\ 换行命令的定义,
\cmd{array\-back\-slash} 用来恢复之。}:
\begin{example}
\begin{tabular}
{>{\centering\arraybackslash}p{9em}}
  \hline
  Some center-aligned long text. \\ 
  \hline
\end{tabular}
\end{example}

\subsection{列宽}\label{subsec:colwidth}

在控制列宽方面,\LaTeX\ 表格有着明显的不足:\texttt{l/c/r} 格式的列宽是由文字内容的自然宽度决定的,
而 \texttt{p} 格式给定了列宽却不好控制对齐(需要 \pkg{array} 宏包辅助),
更何况列与列之间通常还有间距,所以直接生成给定总宽度的表格并不容易。

\envindex{tabular*}
\LaTeX\ 本身提供了 \env{tabular*} 环境用来排版定宽表格,但是不太方便使用,
比如要用到 \texttt{@} 格式插入额外命令,令单元格之间的间距为 \cmd{fill},但即使这样仍然有瑕疵:
\begin{example}
\begin{tabular*}{14em}%
{@{\extracolsep{\fill}}|c|c|c|c|}
  \hline
  A & B & C & D \\ \hline
  a & b & c & d \\ \hline
\end{tabular*}
\end{example}

\pkgindex{tabularx}
\envindex[tabularx]{tabularx}
\pkg{tabularx} 宏包为我们提供了方便的解决方案。它引入了一个 \texttt{X} 格式,类似 \texttt{p} 格式,
不过会根据表格宽度自动计算列宽,多个 \texttt{X} 格式平均分配列宽。\texttt{X} 格式也可以用 \pkg{array} 里的辅助格式修饰对齐方式:
\begin{example}
\begin{tabularx}{14em}%
{|*{4}{>{\centering\arraybackslash}X|}}
  \hline
  A & B & C & D \\ \hline
  a & b & c & d \\ \hline
\end{tabularx}
\end{example}

\subsection{横线}\label{subsec:hline}

\cmdindex{hline,cline}
我们已经在之前的例子见过许多次绘制表格线的 \cmd{hline} 命令。另外 \cmd{cline}\marg*{\Arg{i}-\Arg{j}} 用来绘制跨越部分单元格的横线:
\begin{example}
\begin{tabular}{|c|c|c|}
  \hline
  4 & 9 & 2 \\ \cline{2-3}
  3 & 5 & 7 \\ \cline{1-1}
  8 & 1 & 6 \\ \hline
\end{tabular}
\end{example}

\pkgindex{booktabs}
\cmdindex[booktabs]{toprule,midrule,bottomrule}
在科技论文排版中广泛应用的表格形式是三线表,形式干净简明。
三线表由 \pkg{booktabs} 宏包支持,它提供了 \cmd{toprule}、\cmd{midrule} 和 \cmd{bottomrule} 命令用以排版三线表的三条线,
除此之外,最好不要用其它横线以及竖线:
\begin{example}
\begin{tabular}{cccc}
  \toprule
           & 1 & 2 & 3 \\
  \midrule
  Alphabet & A & B & C \\
  Roman    & I & II& III \\
  \bottomrule
\end{tabular}
\end{example}

\subsection{合并单元格}\label{subsec:tabular-multicol}

\LaTeX\ 是一行一行排版表格的,横向合并单元格较为容易,由 \cmd{multi\-column} 命令实现:
\begin{command}
\cmd{multicolumn}\marg{n}\marg{column-spec}\marg{item}
\end{command}
其中 \Arg{n} 为要合并的列数,\Arg{column-spec} 为合并单元格后的列格式,只允许出现一个 \texttt{l/c/r} 或 \texttt{p} 格式。
如果合并前的单元格前后带表格线 \texttt|,合并后的列格式也要带 \texttt| 以使得表格的竖线一致。
\begin{example}
\begin{tabular}{|c|c|c|}
  \hline
  1 & 2 & Center \\ \hline
  \multicolumn{2}{|c|}{3} &
  \multicolumn{1}{r|}{Right} \\ \hline
  4 & \multicolumn{2}{c|}{C} \\ \hline
\end{tabular}
\end{example}

上面的例子还体现了,形如 \cmd{multicolumn}\marg*{1}\marg{column-spec}\marg{item} 的命令\textbf{可以用来修改某一个单元格的列格式。}

\pkgindex{multirow}
\cmdindex[multirow]{multirow}
纵向合并单元格需要用到 \pkg{multirow} 宏包提供的 \cmd{multirow} 命令:
\begin{command}
\cmd{multirow}\marg{n}\marg{width}\marg{item}
\end{command}
\Arg{width} 为合并后单元格的宽度,可以填 \texttt{*} 以使用自然宽度。

我们看一个结合 \cmd{cline}、\cmd{multi\-column} 和 \cmd{multi\-row} 命令的例子:
\begin{example}
\begin{tabular}{ccc}
  \hline
  \multirow{2}{*}{Item} & 
    \multicolumn{2}{c}{Value} \\
  \cline{2-3}
    & First & Second \\ \hline
  A & 1     & 2 \\ \hline
\end{tabular}
\end{example}

\subsection{行距控制}\label{subsec:tabular-colht}

\cmdindex{arraystretch}
\LaTeX\ 生成的表格看起来通常比较紧凑。修改参数 \cmd{array\-stretch} 可以得到行距更加宽松的表格
(相关命令参考 \ref{sec:newcommands} 节):
\begin{example}
\renewcommand\arraystretch{1.8}
\begin{tabular}{|c|}
  \hline
  Really loose \\ \hline
  tabular rows.\\ \hline
\end{tabular}
\end{example}

\index{\@\crcmd\ (\textit{换行})}
另一种增加间距的办法是给换行命令 \crcmd\ 添加可选参数,在这一行下面加额外的间距,适合用于在行间不加横线的表格:
\begin{example}
\begin{tabular}{c}
  \hline
  Head lines \\[6pt]
  tabular lines \\
  tabular lines \\ \hline
\end{tabular}
\end{example}

但是这种换行方式的存在导致了一个缺陷——\textbf{表格的首个单元格不能直接使用中括号 \texttt{[]}},
否则 \crcmd\ 往往会将下一行的中括号当作自己的可选参数,因而出错。如果要使用中括号,应当放在花括号 \marg*{} 里面。
或者也可以选择将换行命令写成 \crcmd\texttt{[0pt]}。

\pinyinindex{biaoge}{表格|)}

\section{图片}\label{sec:figures}

\pkgindex{graphicx}

\LaTeX\ 本身不支持插图功能,需要由 \pkg{graphicx} 宏包辅助支持。

使用 \texttt{latex + dvipdfmx} 编译命令时,调用 \pkg{graphicx} 宏包时要给定 \texttt{dvipdfmx} 选项%
\footnote{早期常使用 \texttt{latex + dvips} 组合命令,后者将 \texttt{.dvi} 文件转为 \texttt{.ps} 文件(PostScript),
可进一步通过 ps2pdf 工具生成 PDF。\texttt{dvips} 和 \texttt{dvipdfmx} 在图形、颜色、超链接等功能的实现上有差别,而 \LaTeX\ 无法识别
用户是用 \texttt{dvips} 还是 \texttt{dvipdfmx},所以要给定选项(缺省为 \texttt{dvips})。
\ref{sec:hyperlinks} 节中的 \pkg{hyperref} 宏包同理。};而使用 \texttt{pdflatex} 或 \texttt{xelatex} 命令编译时不需要。

读者可能听说过“\LaTeX\ 只能插入 \texttt{.eps} 格式的图片,需要把 \texttt{.jpg} 转成 \texttt{.eps} 格式”的观点。
\LaTeX\ 发展到今天,这个观点早已过时。事实上不同的编译命令支持的图片格式范围各异,见表 \ref{tbl:figure-format}。
这个表格也能解答诸如“为什么 \texttt{.eps} 格式图片在 \texttt{pdflatex} 编译命令下出错”之类的问题。本表格也再一次说明,使用
\texttt{xelatex} 命令是笔者最推荐的方式。

\begin{table}[htp]
\begingroup\centering
\caption{各种编译方式支持的主流图片格式。}\label{tbl:figure-format}
\begin{tabular}{l>{\ttfamily}l>{\ttfamily}l}
 \hline
 格式  & 矢量图 & 位图 \\
 \hline
 \texttt{latex + dvipdfmx}           & .eps      & n/a \\
 \quad $\llcorner$(调用 \pkg{bmpsize} 宏包 )   & .eps .pdf     & .jpg .png .bmp \\[.3\baselineskip]
 \texttt{pdflatex}                   & .pdf      & .jpg .png \\
 \quad $\llcorner$(调用 \pkg{epstopdf} 宏包)   & .pdf .eps & .jpg .png \\[.3\baselineskip]
 \texttt{xelatex}                    & .pdf .eps & .jpg .png .bmp .tif \\
 \hline
\end{tabular}\par\endgroup
\begin{quotation}\small
注:在较新的 \TeX\ 发行版中,\texttt{latex + dvipdfmx} 和 \texttt{pdf\-latex} 命令可不依赖宏包,支持原来需要宏包扩展的图片格式
(但 \texttt{pdf\-latex} 命令仍不支持除 \texttt{.jpg} 和 \texttt{.png} 以外的位图)。
\end{quotation}
\end{table}

\cmdindex[graphicx]{includegraphics}
在调用了 \pkg{graphicx} 宏包以后,就可以使用 \cmd{include\-graphics} 命令加载图片了:
\begin{command}
\cmd{includegraphics}\oarg{options}\marg{filename}
\end{command}

\cmdindex[graphicx]{graphicspath}
其中 \Arg{filename} 为图片文件名,与使用 \cmd{include} 命令类似,文件名有时需要使用相对路径或绝对路径(见 \ref{sec:latex-multi-files} 节)。
图片文件的扩展名可写可不写。

另外 \pkg{graphicx} 宏包还提供了 \cmd{graphics\-path} 命令,用于声明一个或多个图片文件存放的目录,
使用这些目录里的图片时可不用写路径:
\begin{verbatim}
% 假设主要的图片放在 figures 子目录下,标志放在 logo 子目录下
\graphicspath{{figures/}{logo/}}
\end{verbatim}

\cmd{includegraphics} 命令的可选参数 \Arg{options} 支持 \Arg{key}=\Arg{value} 形式赋值,常用的选项如下:
\begin{center}
\begin{tabular}{lp{18em}}
 \hline
 选项 & 含义 \\
 \hline
 width=\Arg{width}    &  将图片缩放到宽度为 \Arg{width} \\
 height=\Arg{height}  &  将图片缩放到高度为 \Arg{height} \\
 scale=\Arg{scale}    &  将图片相对于原尺寸缩放 \Arg{scale} 倍 \\
 angle=\Arg{angle}    &  令图片逆时针旋转 \Arg{angle} 度 \\
 \hline
\end{tabular}
\end{center}

\section{盒子}\label{sec:box}

\section{浮动体}\label{sec:float}

\endinput