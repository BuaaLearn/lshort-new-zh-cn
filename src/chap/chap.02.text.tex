\chapter{用 \LaTeX\ 排版文字}

\section{语言文字和编码}\label{sec:encoding}

\LaTeX\ 源代码为文本文件,而文本文件的一个至关重要的性质是它的编码。在此用尽量短的篇幅介绍一下。

\subsection{ASCII 编码}\label{subsec:ascii}

计算机的基本存储单位是字节(byte),每个字节为八位(8-bit),能够代表 0-255 这 256 个数。
ASCII (美国通用信息交换码)覆盖前 128 个数 0-127,也就是 7-bit,能够表示 QWERTY 键盘上能够输入的拉丁字母、数字和符号。

由于 \TeX\ 最初面向英语文档的排版,ASCII 编码完全够用,而 \TeX\ 最早也只支持 7-bit 和 ASCII。
如果要排版西欧语系中带重音的字符,就要用到后文所叙述的重音命令了,如 M\"obius 必须输入 \texttt{M\char`\\"obius}。

\subsection{扩展编码}\label{subsec:ext-encoding}

在 ASCII 之后,各种语言都发展出来了自己的编码,比如西欧语言的 Latin-1,日本的 Shift-JIS,中国大陆的 GB2312-80 和 GBK 等。
它们之中的绝大多数都向下兼容 ASCII,因此无论是在哪种编码下,\TeX\ 以及 \LaTeX\ 的命令和符号都能用。

西欧(拉丁字母)、俄语系(西里尔字母)等的编码较容易处理,因为它们刚好能够利用 128-255 这个编码范围。
\texttt{latex} 命令及 \texttt{pdflatex} 命令下,这些编码的支持由 \pkg{inputenc} 宏包提供。比如将文档保存为 Latin-1 编码,并在导言区使用:
\begin{verbatim}
\usepackage[latin1]{inputenc}
\end{verbatim}

接下来,M\"obius 就直接可以通过(用适当输入法)输入 \texttt{M\"obius} 得到了。其它一些语言也可以使用 \pkg{inputenc} 宏包
配合 \pkg{babel} 宏包排版。

GBK 等编码是多字节编码,ASCII 字符为一个字节,而非 ASCII 字符为两个字节,这就需要借助一些宏包进行较为复杂的判断和处理。
\pkg{CJK} 宏包就是处理汉语等多字节编码的语言使用的宏包。但 \pkg{CJK} 宏包的使用非常不方便,笔者不再推荐直接使用。
后一节将会简单介绍现在推荐的中文支持方案。

\subsection{UTF-8 编码}\label{subsec:utf8}

Unicode 是一个多国语言文字的集合,覆盖了几乎全球范围内的语言文字。UTF-8 是 Unicode 的一套编码方案,
一个字符可以由一个到四个(现在用到三个)字节编码,其中单字节范围仍然是 ASCII 字符。

\texttt{latex} 命令及 \texttt{pdflatex} 命令下可以使用 \pkg{inputenc} 宏包支持 UTF-8:
\begin{verbatim}
\usepackage[utf8]{inputenc}
\end{verbatim}

\texttt{xelatex} 命令原生支持 UTF-8 编码,而且也\textbf{不适用 \pkg{inputenc} 宏包}。将源代码以 UTF-8 格式保存,
并用 \pkg{fontspec} 宏包(见 \ref{subsec:fontspec} 小节)载入所需的字体,就可以在 \texttt{.tex} 源代码中输入任意语言的文字。
但各个语言的特殊排版要求需要更多的宏包支持(印地语、阿拉伯语等等),如 \pkg{babel}、 \pkg{polyglossia} 等。

\section{排版中文}

\subsection{\pkg{xeCJK} 宏包}

用 \LaTeX\ 排版中文的一大门槛是中文字体。\TeX\ 使用的字体格式仅支持不超过 256 个字符,要想排版中文,比如旧式的 \pkg{CJK} 宏包,
往往需要复杂的预处理,将中文字体拆分成数百个小字体,非常麻烦%
\footnote{现在的 \texttt{pdflatex} 以及 \texttt{latex} + \texttt{dvipdfmx} 命令通过字体映射的方式支持直接使用中文字体,不需要经过预处理,
但仍然是不方便新手操作的。}。

\hologo{XeLaTeX} 支持直接使用系统安装的 TTF/OTF 等格式的字体,加上对 UTF-8 编码的原生支持,免去了预处理字体的麻烦。
在此基础上的 \pkg{xeCJK} 宏包更进一步完善了中英文混排、标点符号等排版细节,比如中英文之间插入空隙、
中文行尾的回车不引入空格、标点符号不出现在行首,等等。

\pkg{xeCJK} 宏包支持用简单的命令配置中文字体。举一个在 Windows 下使用 \pkg{xeCJK} 的例子,源代码须保存为 UTF-8 编码
(设置字体的命令详见 \ref{subsec:CJKfont} 小节):
\begin{verbatim}
\documentclass{article}
\usepackage{xeCJK}
\setCJKmainfont{SimSun}
\begin{document}
中文LaTeX排版。
\end{document}
\end{verbatim}

\subsection{\pkg{ctex} 宏包和文档类}

\pkg{ctex} 宏包和文档类是在 \pkg{CJK} 和 \pkg{xeCJK} 基础上的进一步封装。\pkg{ctex} 文档类包括
\cls{ctexart} / \cls{ctexrep} / \cls{ctexbook},是对 \LaTeX\ 的三个标准文档类的封装,对 \LaTeX\ 的排版样式做了许多调整,
以切合中文排版风格。最新版本的 \pkg{ctex} 宏包/文档类甚至支持自动配置字体。比如上述例子可进一步简化为:
\begin{verbatim}
\documentclass{ctexart}
\begin{document}
中文LaTeX排版。
\end{document}
\end{verbatim}

\pkg{ctex} 宏包/文档类支持源代码使用 UTF-8 和 GBK 编码(要在文档类指定 UTF8 或 GBK 选项,详见 \pkg{ctex} 宏包参考手册),
用 \texttt{latex} + \texttt{dvipdfmx} 命令、\texttt{pdflatex} 或 \texttt{xelatex} 命令(不支持 GBK)都能够编译。
推荐使用 UTF-8 编码编写源代码,用 \texttt{xelatex} 命令编译。

\section{\LaTeX\ 中的符号}

\subsection{空格和分段}\label{subsec:spaces}

\LaTeX\ 源代码中,空格键和 Tab 键输入的空白字符视为“空格”。
连续的若干个空白字符视为一个空格。另外一行开头的空格是忽略不计的。

行末的回车视为一个空格;但连续两个回车,也就是空行,会将文字分段。多个空行被视为一个空行。
也可以在行末使用 \cmd{par} 命令分段:
\begin{example}
Several spaces     equal one.
  Front spaces are ignored.

An empty line starts a new
paragraph.\par
A \verb|\par| command also 
starts a new line.
\end{example}

\subsection{注释}\label{subsec:comments}

\LaTeX\ 用 \texttt\% 字符作为注释。在这个字符之后直到行末,所有的字符都被忽略,连同回车引入的空格。
我们需要注意以下示例中“回车引入的空格被忽略”的效果:
\begin{example}
This is an % short comment
% ---
% Long and organized
% comments
% ---
example: Comments do not bre%
ak a word.
\end{example}

\subsection{特殊字符}\label{subsec:special-chars}

以下字符在 \LaTeX\ 里有特殊用途,如 \texttt\% 表示注释, \texttt\$、\texttt\textasciicircum 、\texttt\_ 等用于排版数学公式,
\texttt\& 用于排版表格,等等。直接输入这些字符得不到对应的符号,还往往会出错:
\begin{verbatim}
# $ % & { } _ ^ ~ \
\end{verbatim}

如果想要输入以上符号,需要使用以下带反斜线的形式输入:
\begin{example}
\# \$ \% \& \{ \} \_ 
\^{} \~{}
\textbackslash
\end{example}

事实上这些带反斜线的形式就是 \LaTeX\ 命令。\cmd{\textasciicircum} 和 \cmd{\textasciitilde} 
两个命令的形式比较特殊,如果不加一对括号,就和后面的字符形成了重音效果(详见某节)。
\cmd{\char`\\} 被直接定义成了手动换行的命令,输入反斜杠就只好用 \cmd{text\-back\-slash}。

\section{段落}

\leavevmode\nobreakspace

\section{单词之间的空格}

\leavevmode\nobreakspace

\section{断行和断页}

\leavevmode\nobreakspace

\subsection{手动断行和断页}

\leavevmode\nobreakspace

\subsection{连字符}

\leavevmode\nobreakspace

\section{预定义的字符串}

\leavevmode\nobreakspace

\endinput