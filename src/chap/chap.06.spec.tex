\chapter{特色工具和功能}

\begin{intro}
本章介绍一些特色的 \LaTeX\ 辅助功能。前两个功能 \hologo{BibTeX} 和 makeindex 依靠一些辅助程序自动生成参考文献、索引等;
之后的使用颜色、超链接等则令我们生成美观易用的电子文档。
\end{intro}

\section{参考文献和 \hologo{BibTeX} 工具}

\subsection{基本的参考文献和引用}

\pinyinindex{cankaowenxian}{参考文献}
\LaTeX\ 提供的参考文献和引用方式比较原始,需要用户自行书写参考文献列表。不同学术论文对参考文献列表的格式要求不一样,
自行书写是一件极其头疼的事情。相关的命令我们只作最简单的介绍。

\cmdindex{cite}
\LaTeX\ 提供了最基本的 \cmd{cite} 命令用于在正文中引用参考文献:
\begin{command}
\cmd{cite}\marg{citation}
\end{command}

\cmd{cite} 带一个可选参数,为引用的编号后加上额外的内容,如 \cmd{cite}\oarg*{page 22}\marg{pa} 可能得到形如 [13, page 22] 这样的引用。

\envindex{thebibliography}
参考文献由 \env{thebibliography} 环境包裹。每条参考文献由 \cmd{bibitem} 开头,其后是参考文献本身的内容:
\begin{command}
\cmd{bibitem}\oarg{item number}\marg{citation}
\end{command}

其中 \Arg{citation} 是 \cmd{cite} 用以引用的一个标签,类似交叉引用里的 \cmd{label}。
\Arg{item number} 自定义参考文献的序号,如果省略,则按自然排序给定序号。

\env{thebibliography} 环境带一个参数,用以设定 \cmd{cite} 命令生成的引用编号的最大宽度,
如 99 意味着不超过两位数字。通常设定为与参考文献的数目一致。
\env{the\-biblio\-graphy} 环境自动生成不带编号的一章(\cls{report} / \cls{book} 文档类)或一节(\cls{article} 文档类)。

以下为一个使用 \env{the\-biblio\-graphy} 排版参考文献的例子:

\begin{verbatim}
\documentclass{article}
\begin{document}
\section{Introduction}
Partl~\cite{pa} has proposed that \ldots

\begin{thebibliography}{99}
\bibitem{pa} H.~Partl: \emph{German \TeX}, 
  TUGboat Volume~9, Issue~1 (1988)
\end{thebibliography}
\end{document}
\end{verbatim}

\subsection{\hologo{BibTeX} 数据库}

\index{bibtexdb@\protect\hologo{BibTeX} 数据库}
\hologo{BibTeX} 是最为流行的参考文献数据组织格式之一。它的出现让我们摆脱手写参考文献条目的麻烦。我们还可以通过参考文献格式的支持,
让同一份 \hologo{BibTeX} 数据库生成不同格式的参考文献列表。我们首先简单介绍 \hologo{BibTeX} 数据库,之后介绍如何将数据库利用在 \LaTeX\ 中。

\hologo{BibTeX} 数据库以 \texttt{.bib} 作为扩展名,其内容是若干个文献条目,每个条目的格式为:
\begin{command}
\texttt @\Arg{type}\texttt\{\Arg{citation}, \\
\qquad\Arg{key1} = \marg{value1}, \\
\qquad\Arg{key2} = \marg{value2}, \\
\qquad\ldots\\
\texttt\}
\end{command}

其中 \Arg{type} 为参考文献的类型,如 \texttt{article} 为学术论文,\texttt{book} 为书籍,\texttt{in\-collection} 为论文集中的某一篇,等等。
\Arg{citation} 为 \cmd{cite} 命令所用的参考文献的标签。在 \Arg{citation} 之后为条目里的各个数据项,以 \Arg{key} = \marg{value} 的形式组织。

我们在此简单列举学术论文里使用较多的 \hologo{BibTeX} 文献条目格式。所有条目格式参考 \CTAN|biblio/bibtex/base/btxdoc.pdf|。
\begin{description}
  \item[\texttt{article}] 学术论文,必需数据项有 author, title, journal, year; 可选数据项包括 volumn, issue, number, pages, doi 等;
  \item[\texttt{book}] 书籍,必需数据项有 author/editor, title, publisher, year; 可选数据项包括 volumn/number, series, address 等;
  \item[\texttt{incollection}] 论文集中的一篇,必需数据项有 author, title, booktitle, publisher, year; 可选数据项包括 editor, volumn/number, chapter, pages, address 等;
  \item[\texttt{inbook}] 书中的一章,必需数据项有 author/editor, title, chapter/pages, publisher, year; 可选数据项包括 volumn/number, series, address 等。
\end{description}

多数时候,我们无需自己手写 \hologo{BibTeX} 文献条目。从 Google Scholar 或者期刊/数据库的网站上都能够导出 \hologo{BibTeX} 文献条目,
老牌的文献管理软件 EndNote 也支持生成 \hologo{BibTeX} 格式的数据库。开源软件 JabRef 甚至支持 \hologo{BibTeX} 文献条目的导入、导出和管理。

\subsection{使用 \hologo{BibTeX} 排版参考文献}\label{subsec:bibtex-use}

\index{bibtex@\protect\hologo{BibTeX} 工具}
现在我们来看如何利用 \hologo{BibTeX} 数据库生成参考文献和引用。

第一步:我们当然需要一份 \hologo{BibTeX} 数据库,假设起名为 \texttt{books.bib} ,和我们的 \LaTeX\ 源代码\textbf{一般位于同一个目录下}。

\begin{sourcecode}[htp]
\begin{Verbatim}
@book{Lamport1994,
  title = {{\LaTeX:} A Document Preparation System},
  author = {Leslie Lamport},
  publisher = {Addison-Wesley},
  address = {Reading, Massachusetts},
  year = {1994},
  edition = {2nd}
}
@book{Mittelbach2004,
  title = {The {\LaTeX} Companion},
  author = {Frank Mittelbach and Michel Goossens and 
  Johannes Braams and David Carlisle and Chris Rowley},
  publisher = {Addison-Wesley},
  address = {Reading, Massachusetts},
  year = {2004},
  edition = {2nd}
}
\end{Verbatim}
\caption{\hologo{BibTeX} 数据库示例 \texttt{books.bib}。}
\end{sourcecode}

\cmdindex{bibliographystyle}
第二步:在源代码中添加必要的命令。假设源代码的名称为 \texttt{demo.tex}。首先需要的是在\textbf{导言区}设定参考文献的格式:
\begin{command}
\cmd{bibliographystyle}\marg{bst-name}
\end{command}

其中 \Arg{bst-name} 为格式文件的名称。\hologo{BibTeX} 提供了几个预定义的格式,如 plain, unsrt, alpha 等。
一些学术论文的模板会提供自有的格式文件,以 \texttt{.bst} 作为扩展名,同样和 \LaTeX\ 源代码\textbf{放在同一个目录下}。

\cmdindex{cite}
\cmdindex{nocite}
其次就是在正文中引用参考文献了。\hologo{BibTeX} 程序在生成参考文献列表的时候,通常只列出你用 \cmd{cite} 命令引用的那些。
如果需要列出没有被引用的文献,则需要 \cmd{nocite}\marg{citation} 命令;而 \cmd{nocite}\marg*{*} 则让所有未引用的参考文献都列出。

\cmdindex{bibliography}
再次,在你需要列出参考文献的位置,使用 \cmd{biblio\-graphy} 命令代替 \env{the\-biblio\-graphy} 环境:
\begin{command}
\cmd{bibliography}\marg{bib-name}
\end{command}

其中 \Arg{bib-name} 是 \hologo{BibTeX} 数据库的文件名,\textbf{不要带 \texttt{.bib} 扩展名}。

\begin{sourcecode}[htp]
\begin{Verbatim}
\documentclass{article}
\bibliographystyle{plain}
\begin{document}
\section{Some words}
Some excellent books, for example, \cite{Lamport1994} 
and \cite{Mittelbach2004} \ldots

\bibliography{books}
\end{document}
\end{Verbatim}
\caption{利用 \texttt{books.bib} 生成参考文献的源代码 \texttt{demo.tex}。}
\end{sourcecode}

注意:\cmd{bibliographystyle} 和 \cmd{bibliography} 命令缺一不可,没有这两个命令,使用 \hologo{BibTeX} 生成参考文献列表的时候会报错。

第三步:写好了以上两个文件之后,我们就可以开始编译了。
\begin{enumerate}
  \item 首先使用 \texttt{pdflatex} 或 \texttt{xelatex} 等命令编译 \LaTeX\ 源代码 \texttt{demo.tex};
  \item 接下来用 \texttt{bibtex} 命令处理 \texttt{demo.aux} 文件记录的参考文献格式、引用条目等信息。
  \texttt{bibtex} 命令处理完毕后会生成 \texttt{demo.bbl} 文件,内容就是一个 \env{thebibliography} 环境;
  \item 再使用 \texttt{pdflatex} 或 \texttt{xelatex} 等命令把源代码 \texttt{demo.tex} 编译\textbf{两遍},读入参考文献并正确生成引用。
\end{enumerate}

整个过程使用的命令如下(可以略去扩展名):
\begin{verbatim}
pdflatex demo
bibtex demo
pdflatex demo
pdflatex demo
\end{verbatim}

如果使用 \texttt{latex} 和 {dvipdfmx} 命令编译,则 \texttt{dvipdfmx} 命令放在最后,相当于先后使用
\texttt{latex}, \texttt{bibtex}, \texttt{latex}, \texttt{latex}, \texttt{dvipdfmx}。

\subsection{\pkg{natbib} 宏包}

\pkgindex{natbib}
时下许多学术期刊比较喜欢使用人名——年份的引用方式,形如(\emph{Alice et~al.}, 2005)。
\pkg{natbib} 宏包提供了对这种“自然”引用方式的处理。

\cmdindex[natbib]{citep}
\cmdindex[natbib]{citet}
\pkg{natbib} 宏包在正文中支持两种引用方式:
\begin{command}
\cmd{citep}\marg{citation} \\
\cmd{citet}\marg{citation}
\end{command}

它们分别生成形如(\emph{Alice et~al.}, 2005) 和 \emph{Alice et~al.}(2005) 的人名——年份引用。

\pkg{natbib} 宏包同样也支持数字引用,并且支持将引用的序号压缩,例如:
\begin{verbatim}
\usepackage[numbers,sort&compress]{natbib}
\end{verbatim}
用以上选项调用 \pkg{natbib} 宏包后,连续引用多篇文献时,会生成形如 (3-7) 的引用而不是 (3, 4, 5, 6, 7)。

\pkg{natbib} 宏包还有更多选项和用法,比如默认的引用是用小括号包裹的,可以为宏包添加 \texttt{square} 选项改为中括号;
再比如 \cmd{citep} 命令也支持可选参数,为引用前后都添加额外内容。这里不再赘述,请参考 \pkg{natbib} 宏包的帮助文档。

\section{索引和 makeindex 工具}\label{sec:index}

\pinyinindex{suoyin}{索引}
\index{makeindex@makeindex 工具}
书籍和大文档通常用索引来归纳关键词,方便用户查阅。\LaTeX\ 借助配套的 makeindex 程序完成对索引的排版。

\subsection{使用 makeindex 工具的方法}\label{subsec:makeidx}

要使用索引,须经过这么几个步骤(仍设源代码名为 \texttt{demo.tex}):

\pkgindex{makeidx}
\cmdindex[makeidx]{makeindex}
第一步,在 \LaTeX\ 源代码的导言区调用 \pkg{makeidx} 宏包,并使用 \cmd{makeindex} 命令开启索引的收集:
\begin{verbatim}
\usepackage{makeidx}
\makeindex
\end{verbatim}

\cmdindex{index}
\cmdindex[makeidx]{printindex}
第二步,在正文中需要索引的地方使用 \cmd{index} 命令。\cmd{index} 命令的参数写法详见下一小节;
并在需要输出索引的地方(如所有章节之后)使用 \cmd{printindex} 命令。

第三步,编译过程:

\begin{enumerate}
  \item 首先用 \texttt{pdflatex} 等命令编译源代码 \texttt{demo.tex}。编译过程中产生索引记录文件 \texttt{demo.idx};
  \item 用 makeindex 程序处理 \texttt{demo.idx},生成用于排版的索引列表文件 \texttt{demo.ind};
  \item 再次编译源代码 \texttt{demo.tex},正确生成索引列表。
\end{enumerate}

\subsection{索引项的写法}\label{subsec:index-entry}

\cmdindex{index}
添加索引项的命令为:
\begin{command}
\cmd{index}\marg{index entry}
\end{command}

其中 \Arg{index entry} 为索引项,写法由表 \ref{tbl:index-entry} 汇总。其中 \texttt!、\texttt @ 和 \texttt| 
为特殊符号,如果要向索引项直接输出这些符号,需要加前缀 \texttt";而 \texttt" 需要输入两个引号 \texttt{""} 才能输出到索引项。

\begin{table}[tp]
\centering
\caption{索引项的写法列表。}\label{tbl:index-entry}
\begin{tabular}{@{}lll@{}}
  \hline
  \textbf{举例} &\textbf{索引项} &\textbf{备注}\\
  \hline
  \multicolumn{3}{@{}l@{}}{普通索引} \\
  \verb+hello+              & hello, 1             & 普通索引 \\ 
  \hline
  \multicolumn{3}{@{}l@{}}{分级索引,以 \texttt! 分隔,最多支持三级} \\
  \verb+hello+              & hello, 1             & 一级索引 \\ 
  \verb+hello!Peter+        &\hspace*{2ex}Peter, 3 & 二级索引 \\ 
  \verb+hello!Peter!Jack+   &\hspace*{4ex}Jack,  3 & 三级索引 \\ 
  \hline
  \multicolumn{3}{@{}l@{}}{格式化索引,形式为 \Arg{alpha}\texttt @\Arg{format}} \\
  \multicolumn{3}{@{}l@{}}{\Arg{alpha}为纯字母,用来排序} \\
  \multicolumn{3}{@{}l@{}}{\Arg{format}为索引的格式,可以包括 \LaTeX 代码和简单的公式} \\
  \verb+Mobius@M\""obius+   & M\"obius, 2          & 输出重音 \\
  \verb+alpha@$\alpha$+     & $\alpha$, 7          & 输出公式 \\
  \verb+bold@\textbf{bold}+ & \textbf{bold}, 12    & 输出粗体 \\
  \hline
  \multicolumn{3}{@{}l@{}}{页码范围} \\
  \verb+morning|(+          & morning, 6-7         & 范围索引的开头 \\
  \verb+morning|)+          &                      & 范围索引的结尾 \\
  \hline
  \multicolumn{3}{@{}l@{}}{格式化索引页码} \\
  \verb+Jenny|textbf+       & Jenny, \textbf{3}       & 调用 \cmd{textbf} 加粗页码 \\
  \verb+Joe|see{Jenny}+     & Joe, \textit{see} Jenny & 调用 \cmd{see} 生成特殊形式 \\
  \verb+Joe|seealso{Jenny}+ & Joe, \textit{see also} Jenny & 调用 \cmd{seealso} 生成特殊形式 \\
  \hline
\end{tabular}
\end{table}

读者可以钻研一下下面给出的一个较为复杂的,结合多级索引、索引格式、页码格式等的示例。但在自己使用时,
最好还是遵循“简单的就是最好的”原则,尽量使用表 \ref{tbl:index-entry} 中的写法。
\begin{verbatim}
Test index.
\index{Test@\textsf{""Test}|(textbf}
\index{Test@\textsf{""Test}!sub@"|sub"||see{Test}}
\newpage
Test index.
\index{Test@\textsf{""Test}|)textbf}
\end{verbatim}

\section{使用颜色}\label{sec:color}

\pinyinindex{yanse}{颜色}
\LaTeX\ 原生不支持颜色,它依赖 \pkg{color} 宏包或者 \pkg{xcolor} 宏包,在编译过程中给 PDF 输出生成颜色的特殊指令。

\subsection{颜色的表达方式}

\pkgindex{color}
\cmdindex[color/xcolor]{color}
调用 \pkg{color} 宏包后,我们就可以用如下命令切换颜色:
\begin{command}
\cmd{color}\oarg{color-mode}\marg{code} \\
\cmd{color}\marg{color-name}
\end{command}

颜色的表达方式有两种,其一是使用色彩模型和色彩代码,代码用 $0\sim1$ 的数字代表成分的比例。
\pkg{color} 宏包支持 \texttt{rgb}、\texttt{cmyk} 和 \texttt{gray} 模型,\pkg{xcolor} 支持更多的模型如 \texttt{hsb} 等。
\begin{example}
\large\sffamily
{\color[gray]{0.6} 
  60\% 灰色} \\
{\color[rgb]{0,1,1} 
  青色} 
\end{example}

其二是直接使用名称代表颜色,前提是已经定义了颜色名称(没定义的话会报错):
\begin{example}
\large\sffamily
{\color{red} 红色} \\
{\color{blue} 蓝色} 
\end{example}

\pkgindex{xcolor}
\pkg{color} 宏包仅定义了 8 种颜色名称,\pkg{xcolor} 补充了一些,总共有 19 种,见表 \ref{tbl:colors}。

\def\showcolor#1{%
  \texttt{#1}\index{yanse@颜色!#1@\texttt{#1}}%
  \ \begingroup\fboxsep=0pt\fbox{{\color{#1}\vrule width 1.2em height 1.4ex}}\endgroup}
\def\showxcolor#1{%
  \texttt{#1}\index{yanse@颜色!#1@\texttt{#1} (\pkg{xcolor})}
  \ \begingroup\fboxsep=0pt\fbox{{\color{#1}\vrule width 1.2em height 1.4ex}}\endgroup}

\begin{table}[!hbp]
\centering
\caption{\pkg{color} 和 \pkg{xcolor} 宏包可用的颜色名称。}\label{tbl:colors}
\renewcommand\arraystretch{1.25}
\begin{tabularx}{0.8\textwidth}{*{4}{>{\raggedleft\arraybackslash}X}}
 \hline
 \multicolumn{4}{c}{基本的 8 种颜色名称:} \\
 \showcolor{black} & \showcolor{red} & \showcolor{green} & \showcolor{blue} \\
 \showcolor{white} & \showcolor{cyan} & \showcolor{magenta} & \showcolor{yellow} \\
 \hline
 \multicolumn{4}{c}{\pkg{xcolor} 额外可用的颜色名称:} \\
 \showxcolor{darkgray} & \showxcolor{gray} & \showxcolor{lightgray} &    \\
 \showxcolor{brown}  & \showxcolor{olive} & \showxcolor{orange} & \showxcolor{lime}\\
 \showxcolor{purple} & \showxcolor{teal} & \showxcolor{violet} & \showxcolor{pink} \\
 \hline
\end{tabularx}
\end{table}

\pkg{xcolor} 还支持将颜色通过表达式混合或互补:
\begin{example}
\large\sffamily
{\color{red!40} 40\% 红色}\\
{\color{blue}蓝色
\color{blue!50!black}蓝黑
\color{black}黑色}\\
{\color{-red}红色的互补色}
\end{example}

我们还可以通过命令自定义颜色名称,注意这里的 \Arg{color-mode} 是必选参数:
\begin{command}
\cmd{definecolor}\marg{color-name}\marg{color-mode}\marg{code}
\end{command}

如果调用 \pkg{color} 或 \pkg{xcolor} 宏包时使用 \texttt{dvipsnames} 选项,就有额外的 68 种颜色名称可用。
\pkg{xcolor} 宏包还支持通过其它选项载入更多颜色名称。限于篇幅不展开介绍,详情请参考 \pkg{xcolor} 宏包的手册。

\subsection{带颜色的文本和盒子}

原始的 \cmd{color} 命令类似于字体命令 \cmd{bfseries},它使之后排版的内容全部变成指定的颜色,
所以直接使用时通常要加花括号分组。\pkg{color} / \pkg{xcolor} 宏包都定义了一些方便用户使用的带颜色元素。

\cmdindex[color/xcolor]{textcolor}
输入带颜色的文本可以用类似 \cmd{textbf} 的命令:
\begin{command}
\cmd{textcolor}\oarg{color-mode}\marg{code}\marg{text} \\
\cmd{textcolor}\marg{color-name}\marg{text}
\end{command}

\cmdindex[color/xcolor]{colorbox}
以下命令构造一个带背景色的盒子,\Arg{material} 为盒子中的内容:
\begin{command}
\cmd{colorbox}\oarg{color-mode}\marg{code}\marg{material} \\
\cmd{colorbox}\marg{color-name}\marg{material}
\end{command}

\cmdindex[color/xcolor]{fcolorbox}
以下命令构造一个带背景色和有色边框的盒子,\Arg{fcode} 或 \Arg{fcolor-name} 用于设置边框颜色:
\begin{command}
\cmd{fcolorbox}\oarg{color-mode}\marg{fcode}\marg{code}\marg{material} \\
\cmd{fcolorbox}\marg{fcolor-name}\marg{color-name}\marg{material}
\end{command}

\begin{example}
\sffamily
文字用\textcolor{red}{红色}强调\\
\colorbox[gray]{0.95}{浅灰色背景} \\
\fcolorbox{blue}{yellow}{%
 \textcolor{blue}{蓝色边框+文字,%
  黄色背景}
}
\end{example}

\section{使用超链接}\label{sec:hyperlinks}

\pinyinindex{chaolianjie}{超链接}
PDF 文档格式是现今最流行的电子文档格式,而电子文档最实用的功能之一就是链接功能。
\LaTeX\ 中实现这一功能的 \pkg{hyperref} 宏包堪称 \LaTeX\ 里最强大也最复杂的宏包。

\subsection{\pkg{hyperref} 宏包}\label{subsec:hyperref}

\pkgindex{hyperref}
\pkg{hyperref} 宏包涉及到的链接遍布 \LaTeX\ 的每一个角落——目录、引用、脚注、索引、参考文献等等都被封装成链接。
但这也使得它与其它宏包的冲突机会大大增加,虽然宏包已经尽力解决各方面的兼容性,但仍不能面面俱到。
为减少冲突的可能性,习惯上将 \pkg{hyperref} 宏包\textbf{放在其它宏包之后调用}。

与 \pkg{graphicx} 宏包类似,\texttt{latex} + \texttt{dvipdfmx} 编译方式下,调用 \pkg{hyperref} 宏包时需显式指定 \texttt{dvipdfmx} 选项:
\begin{verbatim}
\usepackage[dvipdfmx]{hyperref}
\end{verbatim}
而 \texttt{pdflatex} 或 \texttt{xelatex} 命令下,直接可用 \cmd{use\-package}\marg*{hyper\-ref} 调用宏包。

\cmdindex[hyperref]{hypersetup}
\pkg{hyperref} 宏包提供了命令 \cmd{hypersetup} 配置各种选项,或者也可以作为宏包选项,在调用宏包时使用:
\begin{command}
\cmd{hypersetup}\marg*{\Arg{option1},\Arg{option2}=\marg*{value},\ldots} \\
\cmd{usepackage}\oarg*{\Arg{option1},\Arg{option2}=\marg*{value},\ldots}\marg*{hyperref}
\end{command}

\subsection{超链接}\label{subsec:url-href}

\cmdindex[hyperref]{url}
\cmdindex[hyperref]{nolinkurl}
\pkg{hyperref} 宏包提供了直接书写超链接的命令,用于在 PDF 中生成 URL:
\begin{command}
\cmd{url}\marg{url} \\
\cmd{nolinkurl}\marg{url}
\end{command}

其中 \cmd{url} 生成可以点击的 URL,而 \cmd{nolinkurl} 只生成文本。在 \cmd{url} 命令中作为参数的 URL 里,
\textbf{可直接输入如 \texttt\%、\texttt\& 这样的特殊符号}。

\cmdindex[hyperref]{href}
我们也可以像网页一样,把一段文字赋予其“超链接”的作用:
\begin{command}
\cmd{href}\marg{url}\marg{text}
\end{command}

\begin{example}
\url{http://wikipedia.org} \\
\nolinkurl{http://wikipedia.org} \\
\href{http://wikipedia.org}{Wiki}
\end{example}

使用 \pkg{hyperref} 宏包后,文档中所有的引用、参考文献、索引等等都转换为超链接。
用户也可将某个交叉引用的标签 \Arg{label} 作为超链接(注意这里的 \Arg{label} 虽然用了方括号,但是是\textbf{必填的}%
\footnote{不带方括号的 \cmd{hyperref} 命令带四个必选参数,前三个组成特殊格式的超链接,最后一个同 \Arg{text}。}):
\begin{command}
\cmd{hyperref}\oarg{label}\marg{text}
\end{command}

默认的超链接在文字外边加上一个带颜色的方框(在打印 PDF 时边框不会打印),可使用 \texttt{color\-links} 选项修改为将文字本身加上颜色,
或修改 \texttt{pdf\-border} 选项的参数;\texttt{hide\-links} 则令超链接既不变色也不加框。
\begin{verbatim}
\hypersetup{hidelinks}
% or:
\hypersetup{pdfborder={0 0 0}}
\end{verbatim}

\subsection{PDF 书签}\label{subsec:pdf-bookmark}

\pinyinindex{PDFshuqian}{PDF 书签}
\pkg{hyperref} 宏包另一个强大的功能是为 PDF 生成书签。对于章节命令 \cmd{chapter}、\cmd{section} 等,
默认情况下会为 PDF 自动生成书签。和交叉引用、索引等类似,生成书签也需要多次编译源代码,第一次编译将书签记录写入 \texttt{.out} 文件,
第二次编译才正确生成书签。

书签的一些属性见表 \ref{tbl:hyperref-settings}。
在 \texttt{latex} + \texttt{dvipdfmx} 或 \texttt{pdflatex} 下使用 \pkg{ctex} 宏包或 \pkg{CJK} 宏包时,
为了正确生成书签而不出现乱码,需要额外的设置,甚至繁琐的工序(这也是我们推荐使用 \texttt{xelatex} 命令处理中文的原因)。

\cmdindex[hyperref]{pdfbookmark}
\pkg{hyperref} 还提供了手动生成书签的命令:
\begin{command}
\cmd{pdfbookmark}\oarg{level}\marg{bookmark}\marg{anchor}
\end{command}
\Arg{bookmark} 为书签名称,\Arg{anchor} 为书签项使用的锚点(类似交叉引用的标签)。可选参数 \Arg{level} 为书签的层级,默认为 0。

\cmdindex[hyperref]{texorpdfstring}
章节命令里往往有 \LaTeX\ 命令甚至数学公式,而 PDF 书签是纯文本,对命令和公式的处理很困难,有出错的风险。
\pkg{hyperref} 宏包已经为我们处理了许多常见命令,如 \cmd{LaTeX} 和字体命令 \cmd{textbf} 等,
其它一些情况就要使用在 \cmd{section} 等命令中使用如下命令,分别提供 \LaTeX\ 代码和 PDF 书签可用的纯文本:
\begin{command}
\cmd{texorpdfstring}\marg{\LaTeX\ code}\marg{PDF bookmark text}
\end{command}
比如:
\begin{verbatim}
\section{\texorpdfstring{$E=mc^2$}{E=mc\textasciicircum 2}}
\end{verbatim}

\subsection{PDF 文档属性}\label{subsec:pdf-settings}

\pkg{hyperref} 宏包还提供了一些选项用于改变 PDF 文档的属性,部分见表 \ref{tbl:hyperref-settings}。

\begin{table}[!hbp]
\centering
\caption{\pkg{hyperref} 宏包提供的设置。}\label{tbl:hyperref-settings}
\begin{tabular}{lcp{18em}}
 \hline
 选项                         &  默认值         & \multicolumn{1}{c}{含义} \\
 \hline
 \texttt{colorlinks=}\Arg{true\textnormal|false} 
                              & \textit{false}  & 设置为 \textit{true} 为链接文字带颜色,反之加上带颜色的边框 \\
 \texttt{hidelinks}           &                 & 取消链接的颜色和边框 \\
 \texttt{pdfborder=}\marg*{\Arg{n} \Arg{n} \Arg{n}}   
                              &  0 0 1          & 超链接边框设置,设为 0 0 0 可取消边框 \\
 \hline
 \texttt{bookmarks=}\Arg{true\textnormal|false} 
                              & \textit{true}   & 是否生成书签 \\
 \texttt{bookmarksopen=}\Arg{true\textnormal|false} 
                              & \textit{false}  & 是否展开书签 \\
 \texttt{bookmarksnumbered=}\Arg{true\textnormal|false} 
                              & \textit{false}  & 书签是否带章节编号 \\
 \texttt{CJKbookmarks=}\Arg{true\textnormal|false} 
                              & \textit{false}   & 使用 \pkg{CJK} 宏包/ \pkg{GBK} 编码时必须的选项,
                                                 在第一次编译后需要将生成的 \texttt{.out} 文件用工具处理编码 \\
 \texttt{unicode}             &                  & 使用 \pkg{CJKutf8} 宏包/ \pkg{UTF-8} 编码时必须的选项 \\
 \hline
 \texttt{pdftitle=}\Arg{string} 
                              & 空               & 标题 \\
 \texttt{pdfauthor=}\Arg{string} 
                              & 空               & 作者 \\
 \texttt{pdfsubject=}\Arg{string} 
                              & 空               & 主题 \\
 \texttt{pdfkeywords=}\Arg{string} 
                              & 空               & 关键词 \\
 \texttt{pdfstartview=}\Arg{Fit\textnormal|FitH\textnormal|FitV}
                              & \textit{Fit}     & 设置 PDF 页面以适合页面/适合宽度/适合高度等方式显示,默认为适合页面 \\
 \hline
\end{tabular}
\end{table}

\endinput