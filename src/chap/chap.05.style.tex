\chapter{排版样式设定}\label{chap:styles}
\addtocontents{los}{\protect\addvspace{10pt}}

\begin{intro}
至此你已经基本学会排版内容丰富的文档,标题、目录、章节、公式、列表、图片、表格等等应有尽有。但是你可能已经有点不甘心了,
因为似乎你排版出来的文档是千篇一律的模样——\LaTeX\ 默认的字体、单调的页眉页脚、不太令你满意的页边距,等等。
本章的内容将带你一览如何修改 \LaTeX\ 的排版样式。
\end{intro}

\section{字体和字号}\label{sec:font}

\LaTeX\ 根据文档的逻辑结构(章节、脚注等)来选择默认的字体样式以及字号。
需要更改字体样式或字号的话,可以使用表 \ref{tbl:fonts} 和表 \ref{tbl:sizes} 中列出的命令。
\begin{example}
{\small The small and
\textbf{bold} Romans ruled}
{\Large all of great big
{\itshape Italy}.}
\end{example}

\LaTeXe\ (相比于早期的 \LaTeX\ 2.09 版本)的一个重要特征是:字体的各种属性是相互独立的,
这意味着用户可以改变字体的大小,而仍然保留字体原有的粗体或者斜体的特性。

\subsection{字体样式}\label{subsec:fontshape}

\pinyinindex{ziti}{字体}
\pinyinindex{fenzu}{分组}
\LaTeX\ 提供了两组修改字体的命令,见表 \ref{tbl:fonts}。其中诸如 \cmd{bfseries} 形式的命令将会影响之后所有的字符,
如果想要让它在局部生效,需要用花括号\textbf{分组},也就是写成 \marg*{\cmd{bfseries}\ \Arg{some text}} 这样的形式;
对应的 \cmd{textbf} 形式带一个参数,只改变参数内部的字体,更为常用。

在公式中,直接使用 \cmd{textbf} 等命令不会起效,甚至报错。\LaTeX\ 已有修改数学字体的命令,详见 \ref{subsec:math-alpha} 小节。

\begin{table}[htp]
\caption{字体命令。} \label{tbl:fonts}
\centering
\begin{tabular}{@{}rrcc@{}}
\hline
\cmd{rmfamily}\cmdindex{rmfamily} & \cmd{textrm}\cmdindex{textrm}\marg*{\ldots}   & \textrm{roman}      & 衬线字体(罗马体)\\
\cmd{sffamily}\cmdindex{sffamily} & \cmd{textsf}\cmdindex{textsf}\marg*{\ldots}   & \textsf{sans serif} & 无衬线字体        \\
\cmd{ttfamily}\cmdindex{ttfamily} & \cmd{texttt}\cmdindex{texttt}\marg*{\ldots}   & \texttt{typewriter} & 等宽字体          \\[\medskipamount]
\cmd{mdseries}\cmdindex{mdseries} & \cmd{textmd}\cmdindex{textmd}\marg*{\ldots}   & \textrm{medium}     & 正常粗细(中等)  \\
\cmd{bfseries}\cmdindex{bfseries} & \cmd{textbf}\cmdindex{textbf}\marg*{\ldots}   & \textbf{bold face}  & 粗体              \\[\medskipamount]
\cmd{upshape}\cmdindex{upshape}   & \cmd{textup}\cmdindex{textup}\marg*{\ldots}   & \textup{upright}    & 直立体            \\
\cmd{itshape}\cmdindex{itshape}   & \cmd{textit}\cmdindex{textit}\marg*{\ldots}   & \textit{italic}     & 意大利斜体        \\
\cmd{slshape}\cmdindex{slshape}   & \cmd{textsl}\cmdindex{textsl}\marg*{\ldots}   & \textsl{slanted}    & 倾斜体            \\
\cmd{scshape}\cmdindex{scshape}   & \cmd{textsc}\cmdindex{textsc}\marg*{\ldots}   & \textsc{Small Caps} & 小字母大写        \\[\medskipamount]
\cmd{em}\cmdindex{em}             & \cmd{emph}\cmdindex{emph}\marg*{\ldots}     & \emph{emphasized}   & 强调,默认斜体    \\
\cmd{normalfont}\cmdindex{normalfont}  & \cmd{textnormal}\cmdindex{textnormal}\marg*{\ldots}   & \textnormal{normal font} & 默认字体 \\
\hline
\end{tabular}
\end{table}

\subsection{字号}\label{subsec:fontsize}

\pinyinindex{zihao}{字号}
字号命令实际大小依赖于所使用的文档类及其选项。表 \ref{tbl:ptsizes} 列出了这些命令在标准文档类中的绝对大小,单位为 pt。

\begin{table}[htp]
\centering
\caption{字号。} \label{tbl:sizes}
\begin{tabular}{@{}ll}
\hline
\cmd{tiny}\cmdindex{tiny}         & \tiny        tiny font \\
\cmd{scriptsize}\cmdindex{scriptsize}   & \scriptsize  very small font\\
\cmd{footnotesize}\cmdindex{footnotesize} & \footnotesize  quite small font \\
\cmd{small}\cmdindex{small}        &  \small            small font \\
\cmd{normalsize}\cmdindex{normalsize}   &  \normalsize  normal font \\
\cmd{large}\cmdindex{large}        &  \large       large font \\
\hline
\end{tabular}%
\qquad\begin{tabular}{ll@{}}
\hline
\cmd{Large}\cmdindex{Large}        &  \Large       larger font \\[5pt]
\cmd{LARGE}\cmdindex{LARGE}        &  \LARGE       very large font \\[5pt]
\cmd{huge}\cmdindex{huge}         &  \huge        huge \\[5pt]
\cmd{Huge}\cmdindex{Huge}         &  \Huge        largest \\
\hline
\end{tabular}
\end{table}

\begin{table}[htp]
\centering
\caption{标准文档类中的字号大小。}\label{tbl:ptsizes}
\begin{tabular}{lrrr}
\hline
字号 & 10pt 选项(默认)& 11pt 选项 & 12pt 选项 \\
\cmd{tiny}\cmdindex{tiny}       & 5pt  & 6pt & 6pt\\
\cmd{scriptsize}\cmdindex{scriptsize} & 7pt  & 8pt & 8pt\\
\cmd{footnotesize}\cmdindex{footnotesize} & 8pt & 9pt & 10pt \\
\cmd{small}\cmdindex{small}        & 9pt & 10pt & 10.95pt \\
\cmd{normalsize}\cmdindex{normalsize} & 10pt & 10.95pt & 12pt \\
\cmd{large}\cmdindex{large}      & 12pt & 12pt & 14.4pt \\
\cmd{Large}\cmdindex{Large}      & 14.4pt & 14.4pt & 17.28pt \\
\cmd{LARGE}\cmdindex{LARGE}      & 17.28pt & 17.28pt & 20.74pt\\
\cmd{huge}\cmdindex{huge}       & 20.74pt & 20.74pt & 24.88pt\\
\cmd{Huge}\cmdindex{Huge}       & 24.88pt & 24.88pt & 24.88pt\\
\hline
\end{tabular}
\end{table}

使用字号命令的时候,通常也需要用花括号进行分组,如同 \cmd{rmfamily} 那样。
\begin{example}
He likes {\LARGE large and
{\small small} letters}.
\end{example}

\cmdindex{fontsize}
\LaTeX\ 还提供了一个基础的命令 \cmd{fontsize} 用于设定任意大小的字号:
\begin{command}
\cmd{fontsize}\marg{size}\marg{base line-skip}
\end{command}

\cmdindex{selectfont}
\cmd{fontsize} 用到两个参数,\Arg{size} 为字号,\Arg{base line-skip} 为基础行距。
表 \ref{tbl:ptsizes} 中的命令也都各自设定了与字号对应的基础行距,大小为字号的 1.2 倍。
如果不是在导言区,\cmd{fontsize} 的设定需要 \cmd{selectfont} 命令才能立即生效,
而表 \ref{tbl:sizes} 的字号设定都是立即生效的。

\LaTeX\ 排版用到的一些老式字体宏包适用表 \ref{tbl:ptsizes} 给出的固定字号%
\footnote{不难看出表 \ref{tbl:ptsizes} 中的许多字号大致呈等比数列,比例为 1.2;而10.95pt实际上是$10\times\sqrt{1.2}$。},
在使用任意的字号大小时往往会报一些警告。如果可能的话,应当尽量使用表 \ref{tbl:sizes} 中的命令设置字号。

\subsection{选用字体宏包}\label{subsec:font-pkgs}

尽管到了这里你知道了如何切换粗体、斜体等等,以及如何改变字号,
但你依然用着 \LaTeX\ 默认的那套、由高德纳设计制作的 Computer Modern 字体。
有的人可能很喜欢 Times、Palatino,或者更好看的字体。这些字体样式的自由设置在 \LaTeX\ 里还不太容易。

幸好大部分时候,许多字体宏包为我们完成了整套配置,我们可以在调用宏包之后,照常使用 \cmd{bfseries} 或 \cmd{ttfamily} 等我们熟悉的命令。
表 \ref{tbl:font-pkgs} 列出了较为常用的字体宏包,其中相当多的宏包还配置了数学字体,或者文本、数学字体兼而有之。
更多的的字体配置参考 \cite{survey,fontcatalogue}。

\subsection{字体编码}\label{subsec:font-encs}

字体编码对于 \LaTeX\ 用户来讲是一个比较难懂的概念。它指定了一个字体里面包含了哪些符号、符号如何编码等等细节。
需要明确一点:字体编码并不与我们在 \ref{subsec:ascii} 等小节叙述的 ASCII 编码等一一对应。

常见的正文字体编码有 \texttt{OT1} 和 \texttt{T1} 等。\LaTeX\ 默认使用对原始 \TeX\ 兼容的 \texttt{OT1} 编码,使用起来有诸多限制:
高德纳在设计 Computer Modern 字体时认为一些符号,如大于号、小于号等,原则上都应该在公式里出现,所以在正文字体(\cmd{rmfamily} 或 \cmd{sffamily})里,
这些符号所在的位置被其它符号所占据(事实上用户输入 \texttt< 和 \texttt> 得到的是\ !` 和\ ?` 两个倒立的标点符号,
正常的大于号和小于号可用命令 \cmd{textgreater} 和 \cmd{textless} 输入;\cmd{ttfamily} 字体下基本上是正常的)。

扩展的 \texttt{T1} 编码则对 ASCII 字符的兼容好得多,不会出现上述的大于号、小于号的问题。
\texttt{T1} 编码配合一些字体宏包如 \pkg{txfonts}、\pkg{lmodern} 等,还能够令用户使用 \cmd{textasciitilde} 命令
输入位置居中的连字符 a\textasciitilde b,相比数学符号 \texttt\$\cmd{sim}\texttt\$ 来得合理一些。

\pkgindex{fontenc}
切换字体编码要用到 \pkg{fontenc} 宏包:
\begin{verbatim}
\usepackage[T1]{fontenc}
\end{verbatim}

\begin{table}[!p]
\centering
\caption{常见的 \LaTeX\ 字体宏包。}\label{tbl:font-pkgs}
\begin{tabular*}{\linewidth}{@{\extracolsep{\fill}}cp{0.65\linewidth}@{}}
 \hline
 \multicolumn{2}{c}{文本/数学字体搭配的宏包} \\
 \hline
 \pkg{lmodern}     & Latin Modern 字体,对 Computer Modern 字体的扩展  \\
 \pkg{cmbright}    & 仿 Computer Modern 风格的无衬线字体 \\
 \pkg{euler}       & Euler 风格数学字体,也出自于高德纳之手 \\
 \pkg{ccfonts}     & Concrete 风格字体 \\
 \pkg{txfonts}     & Times 风格的字体宏包  \\
 \pkg{pxfonts}     & Palatino 风格的字体宏包  \\
 \pkg{stix}        & Times 风格的字体宏包  \\
 \pkg{newtxtext},\pkg{newtxmath}  & \pkg{txfonts} 的改进版本,分别设置文本和数学字体  \\
 \pkg{newpxtext},\pkg{newpxmath}  & \pkg{pxfonts} 的改进版本,分别设置文本和数学字体  \\
 \pkg{mathptmx}    & \pkg{psnfss} 组件之一,Times 风格,较为陈旧,不推荐使用  \\
 \pkg{mathpazo}    & \pkg{psnfss} 组件之一,Palatino 风格,较为陈旧,不推荐使用  \\
 \pkg{fourier}     & fourier 风格数学字体,配合 Utopia 正文字体 \\
 \pkg{fouriernc}   & fourier 风格数学字体,配合 New Century Schoolbook 正文字体 \\
 \pkg{arev}        & Arev 无衬线字体宏包,Bitstream Vera sans 风格 \\
 \pkg{mathdesign}  & 配合 Charter/Garamond/Utopia 正文字体的数学字体宏包(Garamond 字体可能需要单独安装) \\
 \hline
 \multicolumn{2}{c}{文本字体宏包} \\
 \hline
 \pkg{dejavu}      & DejaVu 开源字体 \\
 \pkg{droid}       & Droid 开源字体 \\
 \pkg{inconsolata} & Inconsolata 开源等宽字体 \\
 \pkg{libertine}   & Linux Libertine 衬线字体 \\
 \pkg{sourcesanspro} & Source Sans Pro 开源无衬线字体 \\
 \pkg{sourcecodepro} & Source Code Pro 开源等宽字体 \\
 \hline
 \multicolumn{2}{c}{符号宏包} \\
 \hline
 \pkg{mathabx}     & 数学符号宏包之一 \\
 \pkg{MnSymbol}    & 数学符号宏包之一 \\
 \pkg{fdsymbol}    & 数学符号宏包之一 \\
 \pkg{pifont}      & Zapf Dingbats 符号字体宏包 \\
 \hline
\end{tabular*}
\end{table}

\subsection{使用 \pkg{fontspec} 宏包更改字体 (\texttt{xelatex})}\label{subsec:fontspec}

\index{xelatex@\texttt{xelatex} 命令}
\texttt{xelatex} 编译命令能够支持直接调用系统安装的 \texttt{.ttf} 或 \texttt{.otf} 格式字体%
\footnote{Linux 下的 \TeX\ Live 为了支持系统安装的字体,需要额外的配置。详见附录 \ref{app:install}。}。相比于上一小节,我们有了更多修改字体的余地。

\pkgindex{fontspec}
\cmdindex[fontspec]{setmainfont,setsansfont,setmonofont}
\texttt{xelatex} 命令下支持用户调用字体的宏包是 \pkg{fontspec}。宏包提供了几个设置全局字体的命令,设置 \cmd{rmfamily} 等对应命令的默认字体%
\footnote{新版本 \pkg{fontspec} 的命令支持把必选参数 \Arg{font name} 放在可选参数 \Arg{font features} 的前面。}:
\begin{command}
\cmd{setmainfont}\oarg{font features}\marg{font name} \\
\cmd{setsansfont}\oarg{font features}\marg{font name} \\
\cmd{setmonofont}\oarg{font features}\marg{font name}
\end{command}
其中 \Arg{font name} 使用字体的文件名(带扩展名)或者字体的英文名称。\Arg{font features} 用来手动配置对应的粗体或斜体
,比如为 Windows 下的无衬线字体 Arial 配置粗体和斜体(通常情况下自动检测并设置对应的粗体和斜体,无需手动指定):
\begin{verbatim}
\setsansfont[BoldFont={Arial Bold}, ItalicFont={Arial Italic}]{Arial}
\end{verbatim}
\Arg{font features} 还能配置字体本身的各种特性,这里不再赘述,感兴趣的读者请参考 \pkg{fontspec} 宏包的帮助文档。

需要注意的是:\pkg{fontspec} 宏包会覆盖数学字体设置。需要调用表 \ref{tbl:font-pkgs} 中列出的一些数学字体宏包时,
应当在调用 \pkg{fontspec} 宏包时指定 \texttt{no-math} 选项。\pkg{fontspec} 宏包可能被其它宏包或文档类(如 \pkg{xeCJK}、\pkg{ctex} 文档类)自动调用时,
则在文档开头的 \cmd{document\-class} 命令里指定 \texttt{no-math} 选项。

\subsection{使用 \pkg{xeCJK} 宏包更改中文字体}\label{subsec:CJKfont}

\pkgindex{xeCJK}
\cmdindex[xeCJK]{setCJKmainfont,setCJKsansfont,setCJKmonofont}
前文已经介绍过的 \pkg{xeCJK} 宏包使用了和 \pkg{fontspec} 宏包非常类似的语法设置中文字体:
\begin{command}
\cmd{setCJKmainfont}\oarg{font features}\marg{font name} \\
\cmd{setCJKsansfont}\oarg{font features}\marg{font name} \\
\cmd{setCJKmonofont}\oarg{font features}\marg{font name}
\end{command}

由于中文字体少有对应的粗体或斜体,\Arg{font features} 里多用其他字体来配置,
比如习惯上将宋体的 \texttt{BoldFont} 配置为黑体,而 \texttt{ItalicFont} 配置为楷体。

\section{段落格式和间距}\label{sec:par-lengths}

\subsection{长度和长度变量}\label{subsec:lengths}

在前面的一些章节,我们已经见到一些长度和长度变量的用法。本节首先统一介绍长度和长度变量。

长度的数值 \Arg{length} 由数字和单位组成。常用的单位如下:

\def\unitindex#1{\index{#1@\texttt{#1} (\textit{长度单位})}}

\begin{center}
\begin{tabular}{cl}
 \hline
 \texttt{pt}\unitindex{pt} & 点阵宽度,1/72.27\texttt{in} \\
 \texttt{bp}\unitindex{bp} & 点阵宽度,1/72\texttt{in} \\
 \texttt{in}\unitindex{in} & 英寸 \\
 \texttt{cm}\unitindex{cm} & 厘米 \\
 \texttt{mm}\unitindex{mm} & 毫米 \\
 \hline
 \texttt{em}\unitindex{em} & 当前字号下大写字母 M 的宽度,常用于水平距离的设定 \\
 \texttt{ex}\unitindex{ex} & 当前字号下小写字母 x 的高度,常用于垂直距离的设定 \\
 \hline
\end{tabular}
\end{center}

在一些情况下还会用到可伸缩的“弹性长度”,如 \texttt{12pt plus 2pt minus 3pt} 
表示基础长度为 \texttt{12pt},可以伸展到 \texttt{14pt} ,也可以收缩到 \texttt{9pt}。
也可只定义 \texttt{plus} 或者 \texttt{minus} 的部分,如 \texttt{0pt plus 5pt}。

长度的数值还可以用长度变量本身或其倍数来表达,如 \texttt{2.5}\cmd{parindent} 等。

\cmdindex{newlength,setlength,addtolength}
\LaTeX\ 预定义了大量的长度变量用于控制版面格式。如页面宽度和高度、首行缩进、段落间距等。
如果需要自定义长度变量,需使用如下命令:
\begin{command}
\cmd{newlength}\marg*{\cmd{\Arg{length command}}}
\end{command}

长度变量可以用 \cmd{setlength} 赋值,或用 \cmd{addtolength} 增加长度:
\begin{command}
\cmd{setlength}\marg*{\cmd{\Arg{length command}}}\marg{length} \\
\cmd{addtolength}\marg*{\cmd{\Arg{length command}}}\marg{length}
\end{command}

\subsection{行距}\label{subsec:linespread}

\pinyinindex{hangju}{行距}
\cmdindex{linespread}
前文中我们提到过 \cmd{fontsize} 命令可以为字号设定对应的行距,但我们很少那么用。
更常用的办法是在导言区使用 \cmd{linespread} 命令,
\begin{command}
\cmd{linespread}\marg{factor}
\end{command}

此处的行距,指的是基本行距(相当于 \cmd{fontsize} 命令的第二个参数)而不是字号大小。所以设置 1.5 倍行距的命令 \cmd{line\-spread}\marg*{1.5}
意味着最终行距为 1.8 倍的字号大小。

\cmdindex{selectfont}
如果不是在导言区全局修改,而想要局部地改变某个段落的行距,需要用 \cmd{select\-font} 命令使 \cmd{line\-spread} 命令的改动立即生效:
\begin{example}
{\linespread{2.0}\selectfont
The baseline skip is set to twice
the normal baseline skip. 
Pay attention to the \verb|\par|
command at the end. \par}

In comparison, after the
curly brace has been closed,
everything is back to normal.
\end{example}

\cmdindex{par}
字号的改变是即时生效的,而行距的改变直到文字\textbf{分段}时才生效。
如果你想改变某一部分文字的行距,那么不能简单地将文字包含在花括号内。注意下面两个例子中 \cmd{par} 命令的位置,包括上一个例子的写法
(\cmd{par} 相当于分段,见 \ref{subsec:spaces} 小节):
\begin{example}
{\Large Don't read this!
 It is not true.
 You can believe me!\par}
\end{example}

\begin{example}
{\Large This is not true either.
But remember I am a liar.}\par
\end{example}

\subsection{段落格式}\label{subsec:par-shape}

以下长度分别为段落的左缩进、右缩进和首行缩进:
\begin{command}
\cmd{setlength}\marg*{\cmd{leftskip}}\marg*{20pt}  \\
\cmd{setlength}\marg*{\cmd{rightskip}}\marg*{20pt} \\
\cmd{setlength}\marg*{\cmd{parindent}}\marg*{2em}
\end{command}

它们和设置行距的命令一样,在分段时生效。

\cmdindex{indent,noindent}
\LaTeX\ 默认在段落开始时缩进,长度为你用上述命令设置的 \cmd{parindent}。如果你在某一段不想使用缩进,可使用某一段开头使用
\begin{command}
\cmd{noindent}
\end{command}
命令。相反地,
\begin{command}
\cmd{indent}
\end{command}
命令强制开启一段首行缩进的段落。多个 \cmd{indent} 命令可以累加缩进量。

\pkgindex{indentfirst}
\LaTeX\ 还默认\textbf{在 \cmd{chapter}、\cmd{section} 等章节标题命令之后的第一段不缩进}%
\footnote{\pkg{ctex} 宏包和文档类默认按照中文习惯保持标题后第一段的首行缩进。}。
如果不习惯这种设定,可以调用 \pkg{indent\-first} 宏包:
\begin{verbatim}
\usepackage{indentfirst}
\end{verbatim}

\cmdindex{parskip}
段落间的垂直间距为 \cmd{parskip},如设置段落间距在 \texttt{0.8ex} 到 \texttt{1.5ex} 变动:
\begin{command}
\cmd{setlength}\marg*{\cmd{parskip}}\marg*{1ex plus 0.5ex minus 0.2ex}
\end{command}

\subsection{水平间距}\label{subsec:hspace}

\cmdindex{hspace}
\LaTeX 默认为将单词之间的“空格”转化为水平间距。如果需要在文中手动插入额外的水平间距,可使用如下命令:
\begin{command}
\cmd{hspace}\marg{length}
\end{command}
\begin{example}
This\hspace{1.5cm}is a space
of 1.5 cm.
\end{example}

\cmdindex{hspace*}
\cmd{hspace} 命令生成的水平间距如果位于一行的开头或末尾,则有可能因为断行而被“吞掉”。可使用 \cmd{hspace*} 命令代替 \cmd{hspace} 命令
得到不会因断行而消失的水平间距。

\cmdindex{stretch,fill}
命令 \cmd{stretch}\marg{n} 生成一个特殊弹性长度,参数 \Arg{n} 为权重。它的基础长度为零,但可以无限延伸,直到占满可用的空间。
如果同一行内出现多个 \cmd{stretch}\marg{n},这一行的所有可用空间将按每个 \cmd{stretch} 命令给定的权重 \Arg{n} 进行分配。

命令 \cmd{fill} 相当于 \cmd{stretch}\marg*{1}。

\begin{example}
x\hspace{\stretch{1}}
x\hspace{\stretch{3}}
x\hspace{\fill}x
\end{example}

\cmdindex{quad,qquad}
在正文中用 \cmd{hspace} 命令生成水平间距时,往往使用 \texttt{em} 作为单位,生成的间距随字号大小而变。
我们在数学公式中见过 \cmd{quad} 和 \cmd{qquad} 命令,它们也可以用于文本中,分别相当于 \cmd{hspace}\marg*{1em} 和 \cmd{hspace}\marg*{2em}:

\begin{example}
{\Large big\hspace{1em}y}\\
{\Large big\quad y}\\
nor\hspace{2em}mal\\
nor\qquad mal\\
{\tiny tin\hspace{1em}y}\\
{\tiny tin\quad y}
\end{example}

\subsection{垂直间距}\label{subsec:vspace}

在页面中,段落、章节标题、行间公式、列表、浮动体等元素之间的间距是 \LaTeX\ 预设的。比如 \cmd{parskip} ,默认设置为 \texttt{0pt plus 1pt}。

\cmdindex{vspace}
如果我们想要人为地增加段落之间的垂直间距,可以在两个段落之间的位置使用如下命令:
\begin{command}
\cmd{vspace}\marg{length}
\end{command}

\cmdindex{vspace*}
\cmd{vspace} 的间距在一页的顶端或底端可能被“吞掉”,类似 \cmd{hspace} 在一行的开头和末尾那样。
对应地,\cmd{vspace*} 命令产生不会因断页而消失的垂直间距。\cmd{vspace} 也可用 \cmd{stretch} 设置无限延伸的垂直长度。

\index{\@\crcmd\ (\textit{换行})}
在段落内的两行之间增加垂直间距,一般通过给断行命令 \crcmd\ 加可选参数,如 \crcmd\texttt{[6pt]} 或 \crcmd\texttt{*[6pt]}。
\cmd{vspace} 也可以在段落内使用:
\begin{example}
Use command \verb|\vspace| to
add \vspace{12pt} some spaces
between lines in a paragraph.
\end{example}

\cmdindex{bigskip,medskip,smallskip}
另外 \LaTeX\ 还提供了\cmd{bigskip}, \cmd{medskip}, \cmd{smallskip} 来增加预定义长度的垂直间距。
\begin{example}
\parbox[t]{3em}{TeX\par TeX}
\parbox[t]{3em}{TeX\par\smallskip TeX}
\parbox[t]{3em}{TeX\par\medskip TeX}
\parbox[t]{3em}{TeX\par\bigskip TeX}
\end{example}

\section{页面和分栏}\label{sec:page-columns}

我们不妨回顾一下第一章介绍的文档类属性。\LaTeX\ 允许你通过文档类选项控制纸张的大小(见表 \ref{tbl:ltx-options}),
包括 \texttt{a4paper}、\texttt{letterpaper}等等,并配合字号设置了适合的页边距。

\cmdindex{textheight,textwidth}
控制页边距的参数由图 \ref{fig:layouts} 里给出的各种长度变量控制。
可以用 \cmd{setlength} 命令修改这些长度变量,以达到调节页面尺寸和边距的作用;
反之也可以利用这些长度变量来决定排版内容的尺寸,如在 \env{tabularx} 环境或 \cmd{include\-graphics} 命令的参数里,
设置图片或表格的宽度为 0.8\cmd{textwidth}:

\begin{figure}[!p]
\centering
\layoutpicture*
\caption{本文档的页面参数示意图(奇数页;由 \pkg{layout} 宏包生成)。} \label{fig:layouts}
\end{figure}

但是,如果你想要直接设置页边距等参数,着实是一件麻烦事。我们根据图 \ref{fig:layouts} 将各个方向的页边距计算公式给出(以奇数页为例):
\begin{align*}
\text{\Arg{left-margin}}   &= \text{\ttfamily 1in} 
                            + \text{\cmd{hoffset}}
                            + \text{\cmd{oddsidemargin}} \\
\text{\Arg{right-margin}}  &= \text{\cmd{paperwidth}} 
                            - \text{\Arg{left-margin}}
                            - \text{\cmd{textwidth}} \\
\text{\Arg{top-margin}}    &= \text{\ttfamily 1in} 
                            + \text{\cmd{voffset}}
                            + \text{\cmd{topmargin}}
                            + \text{\cmd{headheight}}
                            + \text{\cmd{headsep}} \\
\text{\Arg{bottom-margin}} &= \text{\cmd{paperheight}}
                            - \text{\Arg{top-margin}}
                            - \text{\cmd{textheight}}
\end{align*}
如果我们想设置合适的 \Arg{left-margin} 和 \Arg{right-margin},就要靠上述方程组把 \cmd{odd\-sidemargin} 和 \cmd{text\-width} 等参数解出来!

幸好 \pkg{geometry} 宏包提供了设置页面参数的简便方法,能够帮我们完成背后繁杂的计算。

\subsection{利用 \pkg{geometry} 宏包设置页面参数}\label{subsec:geometry}

\pkgindex{geometry}
\pinyinindex{xuanxiang}{选项(宏包/文档类)}
\pkg{geometry} 宏包的调用方式类似于 \pkg{graphicx},在 \texttt{latex} + \texttt{dvipdfmx} 命令下需要指定选项 \texttt{dvipdfm}
(注意这里不是 \texttt{dvipdfmx});\texttt{pdflatex} 和 \texttt{xelatex} 编译命令下不需要。

\cmdindex[geometry]{geometry}
你既可以调用 \pkg{geometry} 宏包然后用其提供的 \cmd{geometry} 命令设置页面参数:
\begin{command}
\cmd{usepackage}\marg*{geometry} \\
\cmd{geometry}\marg{geometry-settings}
\end{command}
也可以将参数指定为宏包的选项:
\begin{command}
\cmd{usepackage}\oarg{geometry-settings}\marg*{geometry}
\end{command}

其中 \Arg{geometry-settings} 多以 \Arg{key}=\Arg{value} 的形式组织。

比如,符合 Microsoft Word 习惯的页面设定是 A4 纸张,上下边距 1 英寸,左右边距 1.25 英寸,于是我们可以通过如下两种等效的方式之一设定页边距:
\begin{verbatim}
\usepackage[left=1.25in,right=1.25in,%
  top=1in,bottom=1in]{geometry}
% or like this:
\usepackage[hmargin=1.25in,vmargin=1in]{geometry}
\end{verbatim}

又比如,需要设定周围的边距一致为1.25英寸,可以用更简单的语法:
\begin{verbatim}
\usepackage[margin=1.25in]{geometry}
\end{verbatim}

对于书籍等双面文档,习惯上奇数页右边、偶数页左边留出较多的页边距,而书脊一侧的奇数页左边、偶数页右边页边距较少。我们可以这样设定:
\begin{verbatim}
\usepackage[inner=1in,outer=1.25in]{geometry}
\end{verbatim}

\pkg{geometry} 宏包本身也能够修改纸张大小、页眉页脚高度、边注宽度等等参数。更详细的用法不再赘述,
感兴趣的用户可查阅 \pkg{geometry} 宏包的帮助文档。

\subsection{页面内容的垂直对齐}\label{subsec:raggedbottom}

\LaTeX\ 默认将页面内容在垂直方向分散对齐。对于有大量图表的文档,许多时候想要做到排版匀称的页面很困难,
垂直分散对齐会造成某些页面的垂直间距过宽,还可能报大量的 \texttt{Underfull} \cmd{vbox} 消息。

\cmdindex{raggedbottom}
\LaTeX\ 还提供了另一种策略:将页面内容向顶部对齐,给底部留出高度不一的空白。在导言区或者适合的位置使用
以下命令开启顶部对齐的效果:
\begin{command}
\cmd{raggedbottom}
\end{command}

\cmdindex{flushbottom}
相反地,\cmd{flushbottom} 命令用于设置成默认的分散对齐。

\subsection{分栏}\label{subsec:columns}

\cmdindex{onecolumn,twocolumn}
\LaTeX\ 支持简单的单栏或双栏排版。标准文档类的全局选项 \texttt{onecolumn}、\texttt{twocolumn} 
可控制全文分单栏或双栏排版。\LaTeX\ 也提供了切换单/双栏排版的命令:
\begin{command}
\cmd{onecolumn} \\
\cmd{twocolumn}\oarg{one-column top material}
\end{command}

\cmd{twocolumn} 支持带一个可选参数,用于排版双栏之上的一部分单栏内容。

\cmdindex{newpage,clearpage}
切换单/双栏排版时总是会另起一页(\cmd{clearpage})。
在双栏模式下使用 \cmd{newpage} 会换栏而不是换页;\cmd{clearpage} 则能够换页。

\pkgindex{multicol}
\envindex[multicol]{multicols}
一个比较好用的分栏解决方案是 \pkg{multicol},它提供了简单的 \env{multicols} 环境
(注意不要写成 \env{multicol} 环境)自动产生分栏,如以下环境将内容分为 3 栏:
\begin{verbatim}
\begin{multicols}{3}
...
\end{multicols}
\end{verbatim}

\pinyinindex{fudongti}{浮动体}
\envindex{table,figure}
\pkgindex{float}
\pkg{multicol} 宏包能够在一页之中切换单栏/多栏,也能处理跨页的分栏,且各栏的高度分布平衡。但代价是%
\textbf{在 \env{multicols} 环境中无法正常使用 \env{table} 和 \env{figure} 等浮动体环境},它会直接让浮动体丢失。
\env{multicols} 环境中只能用跨栏的 \env{table*} 和 \env{figure*} 环境,或者用 \pkg{float} 宏包提供的 \texttt{H} 选项固定浮动体的位置。

\section{页眉页脚}\label{sec:pagestyle}

\subsection{基本的页眉页脚样式}\label{subsec:basic-pagesyle}

\cmdindex{pagestyle,thispagestyle}
\pinyinindex{yemei}{页眉}
\pinyinindex{yejiao}{页脚}
\LaTeX\ 中提供了命令 \cmd{pagestyle} 来修改页眉页脚的样式:
\begin{command}
\cmd{pagestyle}\marg{page-style}
\end{command}
另外一个命令只影响当页的页眉页脚样式:
\begin{command}
\cmd{thispagestyle}\marg{page-style}
\end{command}

\Arg{page-style} 参数为样式的名称,在 \LaTeX\ 里预定义了四类样式,见表 \ref{tbl:pagestyle}。

\begin{table}[htp]
\centering
\caption{\LaTeX\ 预定义的页眉页脚样式}\label{tbl:pagestyle}
\begin{tabularx}{0.8\textwidth}{lX}
 \hline
 \texttt{empty}  & 页眉页脚为空 \\
 \texttt{plain}  & 页眉为空,页脚为页码。(article 和 report 文档类默认;book 文档类的每章第一页也为 plain 格式) \\
 \hline
 \texttt{headings}  & 页眉为章节标题和页码,页脚为空。(book 文档类默认) \\
 \texttt{myheadings}  & 页眉为页码及 \cmd{markboth} 和 \cmd{markright} 命令手动指定的内容,页脚为空。\\
 \hline
\end{tabularx}
\end{table}

\clsindex{article,report,book}
其中 \texttt{headings} 的情况较为复杂:
\begin{description}
  \item[\cls{article} 文档类,\texttt{twoside} 选项:] 偶数页为页码和节标题,奇数页为小节标题和页码;
  \item[\cls{article} 文档类,\texttt{oneside} 选项:] 页眉为节标题和页码;
  \item[\cls{book/report} 文档类,\texttt{twoside} 选项:] 偶数页为页码和章标题,奇数页为节标题和页码;
  \item[\cls{book/report} 文档类,\texttt{oneside} 选项:] 页眉为章标题和页码。
\end{description}

\subsection{手动更改页眉页脚的内容}\label{subsec:marks}

\cmdindex{markright,markboth}
对于 headings 或者 myheadings 样式,\LaTeX\ 允许用户使用命令手动修改页眉上面的内容,
特别是因为使用了 \cmd{chapter*} 等命令而无法自动生成页眉页脚的情况:
\begin{command}
\cmd{markright}\marg{right-mark}\\
\cmd{markboth}\marg{left-mark}\marg{right-mark}
\end{command}

在双面排版、\texttt{headings / myheadings} 页眉页脚样式下,\Arg{left-mark} 和 \Arg{right-mark} 的内容分别预期出现在左页(偶数页)和右页(奇数页)。

事实上 \cmd{chapter}、\cmd{section} 等命令内部也使用 \cmd{mark\-both} 或者 \cmd{mark\-right} 写页眉。
\LaTeX\ 默认将页眉的内容都转为大写字母。如果你不喜欢这样,可以尝试以下代码
(相关命令的用法参照 \ref{sec:newcommands} 节)%
\footnote{但是这不能改变页眉的斜体样式(\cmd{slshape}),斜体是定义在 \texttt{headings} 样式里的。
如果不喜欢斜体,可在 \cmd{mark\-both} 等命令的参数里先使用 \cmd{normal\-font} ,再使用想要的字体样式命令,
或直接尝试使用 \pkg{fancyhdr} 宏包。}:
\begin{verbatim}
\renewcommand\chaptermark[1]{%
  \markboth{Chapter \thechapter\quad #1}{}}
\renewcommand\sectionmark[1]{%
  \markright{\thesection\quad #1}}
\end{verbatim}

其中 \cmd{thechapter}、\cmd{thesection} 等命令为章节计数器的数值(详见 \ref{sec:counters} 节)。以上代码适用于 \cls{report/book} 文档类。
对于 \cls{article} 文档类,与两个页眉相关的命令分别为 \cmd{sec\-tion\-mark} 和 \cmd{sub\-sec\-tion\-mark} 。

\subsection{\pkg{fancyhdr} 宏包}\label{subsec:fancyhdr}

\pkgindex{fancyhdr}
\pkg{fancyhdr} 宏包改善了页眉页脚样式的定义方式,允许我们将内容自由安置在页眉和页脚的左、中、右三个位置,还为页眉和页脚各加了一条横线。

\pkg{fancyhdr} 自定义了样式名称 fancy。使用 \pkg{fancyhdr} 宏包定义页眉页脚之前,通常先用 \cmd{page\-style}\marg*{fancy} 调用这个样式。
在 \pkg{fancyhdr} 中定义页眉页脚的命令为:
\begin{command}
\cmd{fancyhead}\oarg{position}\marg*{\ldots}\\
\cmd{fancyfoot}\oarg{position}\marg*{\ldots}
\end{command}
其中 \Arg{position} 为 L(左)/C(中)/R(右) 以及与 O(奇数页)/E(偶数页)字母的组合。

我们用一个示例说明 \pkg{fancyhdr} 的用法,这段代码可以用于导言区,它的效果为将章节标题放在和 headings 一致的位置,但使用加粗格式;
页码都放在页脚正中;修改横线宽度,“去掉”页脚的横线。更多用法请参考 \pkg{fancyhdr} 宏包的帮助文档。

\begin{sourcecode}[hbp]
\begin{Verbatim}
% 导言区部分
\usepackage{fancyhdr}
\pagestyle{fancy}
\renewcommand{\chaptermark}[1]{\markboth{#1}{}}
\renewcommand{\sectionmark}[1]{\markright{\thesection\ #1}}
\fancyhf{} % 清空当前的页眉页脚
\fancyfoot[C]{\bfseries\thepage}
\fancyhead[LO]{\bfseries\rightmark}
\fancyhead[RE]{\bfseries\leftmark}
\renewcommand{\headrulewidth}{0.4pt} % 注意不用 \setlength
\renewcommand{\footrulewidth}{0pt}
\end{Verbatim}
\caption{\pkg{fancyhdr} 宏包的使用方法示例。}\label{code:fancyhdr}
\end{sourcecode}

\endinput